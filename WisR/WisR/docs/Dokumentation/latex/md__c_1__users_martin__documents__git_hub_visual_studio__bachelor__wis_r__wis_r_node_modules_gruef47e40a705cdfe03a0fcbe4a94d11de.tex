\href{https://travis-ci.org/isaacs/rimraf}{\tt !\mbox{[}Build Status\mbox{]}(https\+://travis-\/ci.\+org/isaacs/rimraf.\+svg?branch=master)} \href{https://david-dm.org/isaacs/rimraf}{\tt !\mbox{[}Dependency Status\mbox{]}(https\+://david-\/dm.\+org/isaacs/rimraf.\+svg)} \href{https://david-dm.org/isaacs/rimraf#info=devDependencies}{\tt !\mbox{[}dev\+Dependency Status\mbox{]}(https\+://david-\/dm.\+org/isaacs/rimraf/dev-\/status.\+svg)}

The \href{http://en.wikipedia.org/wiki/Rm_(Unix}{\tt U\+N\+I\+X command}) {\ttfamily rm -\/rf} for node.

Install with {\ttfamily npm install rimraf}, or just drop rimraf.\+js somewhere.

\subsection*{A\+P\+I}

{\ttfamily rimraf(f, callback)}

The callback will be called with an error if there is one. Certain errors are handled for you\+:


\begin{DoxyItemize}
\item Windows\+: {\ttfamily E\+B\+U\+S\+Y} and {\ttfamily E\+N\+O\+T\+E\+M\+P\+T\+Y} -\/ rimraf will back off a maximum of {\ttfamily opts.\+max\+Busy\+Tries} times before giving up, adding 100ms of wait between each attempt. The default {\ttfamily max\+Busy\+Tries} is 3.
\item {\ttfamily E\+N\+O\+E\+N\+T} -\/ If the file doesn\textquotesingle{}t exist, rimraf will return successfully, since your desired outcome is already the case.
\item {\ttfamily E\+M\+F\+I\+L\+E} -\/ Since {\ttfamily readdir} requires opening a file descriptor, it\textquotesingle{}s possible to hit {\ttfamily E\+M\+F\+I\+L\+E} if too many file descriptors are in use. In the sync case, there\textquotesingle{}s nothing to be done for this. But in the async case, rimraf will gradually back off with timeouts up to {\ttfamily opts.\+emfile\+Wait} ms, which defaults to 1000.
\end{DoxyItemize}

\subsection*{rimraf.\+sync}

It can remove stuff synchronously, too. But that\textquotesingle{}s not so good. Use the async A\+P\+I. It\textquotesingle{}s better.

\subsection*{C\+L\+I}

If installed with {\ttfamily npm install rimraf -\/g} it can be used as a global command {\ttfamily rimraf $<$path$>$ \mbox{[}$<$path$>$ ...\mbox{]}} which is useful for cross platform support.

\subsection*{mkdirp}

If you need to create a directory recursively, check out \href{https://github.com/substack/node-mkdirp}{\tt mkdirp}. 