Exposes a basic wrapper on top of \href{https://github.com/isaacs/node-glob}{\tt Glob} / \href{https://github.com/isaacs/minimatch}{\tt minimatch} combo both written by . Glob now uses Java\+Script instead of C++ bindings which makes it usable in Node.\+js 0.\+6.\+x and Windows platforms.

\href{https://nodei.co/npm/fileset/}{\tt !\mbox{[}N\+P\+M\mbox{]}(https\+://nodei.\+co/npm/fileset.\+png?downloads=true\&stars=true)}

Adds multiples patterns matching and exlude ability. This is basically just a sugar A\+P\+I syntax where you can specify a list of includes and optional exclude patterns. It works by setting up the necessary miniglob \char`\"{}fileset\char`\"{} and filtering out the results using minimatch.

$\ast$\href{https://github.com/mklabs/node-fileset/blob/master/CHANGELOG.md#changelog}{\tt Changelog}$\ast$

\subsection*{Install}

\begin{DoxyVerb}npm install fileset
\end{DoxyVerb}


\subsection*{Usage}

Can be used with callback or emitter style.


\begin{DoxyItemize}
\item {\bfseries include}\+: list of glob patterns {\ttfamily foo/$\ast$$\ast$/$\ast$.js $\ast$.md src/lib/$\ast$$\ast$/$\ast$}
\item {\bfseries exclude}\+: {\itshape optional} list of glob patterns to filter include results {\ttfamily foo/$\ast$$\ast$/$\ast$.js $\ast$.md}
\item {\bfseries callback}\+: {\itshape optional} function that gets called with an error if something wrong happend, otherwise null with an array of results
\end{DoxyItemize}

The callback is optional since the fileset method return an instance of Event\+Emitter which emit different events you might use\+:


\begin{DoxyItemize}
\item {\itshape match}\+: Every time a match is found, miniglob emits this event with the pattern.
\item {\itshape include}\+: Emitted each time an include match is found.
\item {\itshape exclude}\+: Emitted each time an exclude match is found and filtered out from the fileset.
\item {\itshape end}\+: Emitted when the matching is finished with all the matches found, optionally filtered by the exclude patterns.
\end{DoxyItemize}

\paragraph*{Callback}


\begin{DoxyCode}
var fileset = require(\textcolor{stringliteral}{'fileset'});

fileset(\textcolor{stringliteral}{'**/*.js'}, \textcolor{stringliteral}{'**.min.js'}, \textcolor{keyword}{function}(err, files) \{
  \textcolor{keywordflow}{if} (err) \textcolor{keywordflow}{return} console.error(err);

  console.log(\textcolor{stringliteral}{'Files: '}, files.length);
  console.log(files);
\});
\end{DoxyCode}


\paragraph*{Event emitter}


\begin{DoxyCode}
var fileset = require(\textcolor{stringliteral}{'fileset'});

fileset(\textcolor{stringliteral}{'**.coffee README.md *.json Cakefile **.js'}, \textcolor{stringliteral}{'node\_modules/**'})
  .on(\textcolor{stringliteral}{'match'}, console.log.bind(console, \textcolor{stringliteral}{'error'}))
  .on(\textcolor{stringliteral}{'include'}, console.log.bind(console, \textcolor{stringliteral}{'includes'}))
  .on(\textcolor{stringliteral}{'exclude'}, console.log.bind(console, \textcolor{stringliteral}{'excludes'}))
  .on(\textcolor{stringliteral}{'end'}, console.log.bind(console, \textcolor{stringliteral}{'end'}));
\end{DoxyCode}


{\ttfamily fileset} returns an instance of Event\+Emitter, with an {\ttfamily includes} property which is the array of Fileset objects (inheriting from {\ttfamily miniglob.\+Miniglob}) that were used during the mathing process, should you want to use them individually.

Check out the \href{https://github.com/mklabs/node-fileset/tree/master/tests}{\tt tests} for more examples.

\subsection*{Sync usage}


\begin{DoxyCode}
var results = fileset.sync(\textcolor{stringliteral}{'*.md *.js'}, \textcolor{stringliteral}{'CHANGELOG.md node\_modules/**/*.md node\_modules/**/*.js'});
\end{DoxyCode}


The behavior should remain the same, although it lacks the last {\ttfamily options} arguments to pass to internal {\ttfamily glob} and {\ttfamily minimatch} dependencies.

\subsection*{Tests}

Run {\ttfamily npm test}

\subsection*{Why}

Mainly for a build tool with cake files, to provide me an easy way to get a list of files by either using glob or path patterns, optionally allowing exclude patterns to filter out the results.

All the magic is happening in \href{https://github.com/isaacs/node-glob}{\tt Glob} and \href{https://github.com/isaacs/minimatch}{\tt minimatch}. Check them out! 