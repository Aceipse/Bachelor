We welcome your contributions! Thanks for helping make Jasmine a better project for everyone. Please review the backlog and discussion lists before starting work. What you\textquotesingle{}re looking for may already have been done. If it hasn\textquotesingle{}t, the community can help make your contribution better.

\subsection*{Links}


\begin{DoxyItemize}
\item \href{http://groups.google.com/group/jasmine-js}{\tt Jasmine Google Group}
\item \href{http://groups.google.com/group/jasmine-js-dev}{\tt Jasmine-\/dev Google Group}
\item \href{https://www.pivotaltracker.com/n/projects/10606}{\tt Jasmine on Pivotal\+Tracker}
\end{DoxyItemize}

\subsection*{General Workflow}

Please submit pull requests via feature branches using the semi-\/standard workflow of\+:


\begin{DoxyCode}
1 git clone git@github.com:yourUserName/jasmine.git              # Clone your fork
2 cd jasmine                                                     # Change directory
3 git remote add upstream https://github.com/jasmine/jasmine.git # Assign original repository to a remote
       named 'upstream'
4 git fetch upstream                                             # Pull in changes not present in your local
       repository
5 git checkout -b my-new-feature                                 # Create your feature branch
6 git commit -am 'Add some feature'                              # Commit your changes
7 git push origin my-new-feature                                 # Push to the branch
\end{DoxyCode}


Once you\textquotesingle{}ve pushed a feature branch to your forked repo, you\textquotesingle{}re ready to open a pull request. We favor pull requests with very small, single commits with a single purpose.

\subsection*{Background}

\subsubsection*{Directory Structure}


\begin{DoxyItemize}
\item {\ttfamily /src} contains all of the source files
\begin{DoxyItemize}
\item {\ttfamily /src/console} -\/ Node.\+js-\/specific files
\item {\ttfamily /src/core} -\/ generic source files
\item {\ttfamily /src/html} -\/ browser-\/specific files
\end{DoxyItemize}
\item {\ttfamily /spec} contains all of the tests
\begin{DoxyItemize}
\item mirrors the source directory
\item there are some additional files
\end{DoxyItemize}
\item {\ttfamily /dist} contains the standalone distributions as zip files
\item {\ttfamily /lib} contains the generated files for distribution as the Jasmine Rubygem and the Python package
\end{DoxyItemize}

\subsubsection*{Self-\/testing}

Note that Jasmine tests itself. The files in {\ttfamily lib} are loaded first, defining the reference {\ttfamily jasmine}. Then the files in {\ttfamily src} are loaded, defining the reference {\ttfamily j\$}. So there are two copies of the code loaded under test.

The tests should always use {\ttfamily j\$} to refer to the objects and functions that are being tested. But the tests can use functions on {\ttfamily jasmine} as needed. {\itshape Be careful how you structure any new test code}. Copy the patterns you see in the existing code -\/ this ensures that the code you\textquotesingle{}re testing is not leaking into the {\ttfamily jasmine} reference and vice-\/versa.

\subsubsection*{{\ttfamily boot.\+js}}

{\bfseries This is new for Jasmine 2.\+0.}

This file does all of the setup necessary for Jasmine to work. It loads all of the code, creates an {\ttfamily Env}, attaches the global functions, and builds the reporter. It also sets up the execution of the {\ttfamily Env} -\/ for browsers this is in {\ttfamily window.\+onload}. While the default in {\ttfamily lib} is appropriate for browsers, projects may wish to customize this file.

For example, for Jasmine development there is a different {\ttfamily dev\+\_\+boot.\+js} for Jasmine development that does more work.

\subsubsection*{Compatibility}


\begin{DoxyItemize}
\item Browser Minimum
\begin{DoxyItemize}
\item I\+E8
\item Firefox 3.\+x
\item Chrome ??
\item Safari 5
\end{DoxyItemize}
\end{DoxyItemize}

\subsection*{Development}

All source code belongs in {\ttfamily src/}. The {\ttfamily core/} directory contains the bulk of Jasmine\textquotesingle{}s functionality. This code should remain browser-\/ and environment-\/agnostic. If your feature or fix cannot be, as mentioned above, please degrade gracefully. Any code that should only be in a non-\/browser environment should live in {\ttfamily src/console/}. Any code that depends on a browser (specifically, it expects {\ttfamily window} to be the global or {\ttfamily document} is present) should live in {\ttfamily src/html/}.

\subsubsection*{Install Dependencies}

Jasmine Core relies on Ruby and Node.\+js.

To install the Ruby dependencies, you will need Ruby, Rubygems, and Bundler available. Then\+: \begin{DoxyVerb}$ bundle
\end{DoxyVerb}


...will install all of the Ruby dependencies. If the ffi gem fails to build its native extensions, you may need to manually install some system dependencies. On Ubuntu\+: \begin{DoxyVerb}$ apt-get install gcc ruby ruby-dev libxml2 libxml2-dev libxslt1-dev
\end{DoxyVerb}


...should get you to the point that {\ttfamily bundle} can install everything.

To install the Node dependencies, you will need Node.\+js, Npm, and \href{http://gruntjs.com/}{\tt Grunt}, the \href{https://github.com/gruntjs/grunt-cli}{\tt grunt-\/cli} and ensure that {\ttfamily grunt} is on your path. \begin{DoxyVerb}$ npm install --local
\end{DoxyVerb}


...will install all of the node modules locally. If when you run \begin{DoxyVerb}$ grunt
\end{DoxyVerb}


...you see that J\+S\+Hint runs your system is ready.

\subsubsection*{How to write new Jasmine code}

Or, How to make a successful pull request


\begin{DoxyItemize}
\item {\itshape Do not change the public interface}. Lots of projects depend on Jasmine and if you aren\textquotesingle{}t careful you\textquotesingle{}ll break them
\item {\itshape Be environment agnostic} -\/ server-\/side developers are just as important as browser developers
\item {\itshape Be browser agnostic} -\/ if you must rely on browser-\/specific functionality, please write it in a way that degrades gracefully
\item {\itshape Write specs} -\/ Jasmine\textquotesingle{}s a testing framework; don\textquotesingle{}t add functionality without test-\/driving it
\item {\itshape Write code in the style of the rest of the repo} -\/ Jasmine should look like a cohesive whole
\item {\itshape Ensure the {\itshape entire} test suite is green} in all the big browsers, Node, and J\+S\+Hint -\/ your contribution shouldn\textquotesingle{}t break Jasmine for other users
\end{DoxyItemize}

Follow these tips and your pull request, patch, or suggestion is much more likely to be integrated.

\subsubsection*{Running Specs}

Jasmine uses the \href{http://github.com/jasmine/jasmine-gem}{\tt Jasmine Ruby gem} to test itself in browser. \begin{DoxyVerb}$ rake jasmine
\end{DoxyVerb}


...and then visit {\ttfamily \href{http://localhost:8888}{\tt http\+://localhost\+:8888}} to run specs.

Jasmine uses the \href{http://github.com/jasmine/jasmine-npm}{\tt Jasmine N\+P\+M package} to test itself in a Node.\+js/npm environment. \begin{DoxyVerb}$ grunt execSpecsInNode
\end{DoxyVerb}


...and then the results will print to the console. All specs run except those that expect a browser (the specs in {\ttfamily spec/html} are ignored).

\subsection*{Before Committing or Submitting a Pull Request}


\begin{DoxyEnumerate}
\item Ensure all specs are green in browser {\itshape and} node
\end{DoxyEnumerate}
\begin{DoxyEnumerate}
\item Ensure J\+S\+Hint is green with {\ttfamily grunt jshint}
\end{DoxyEnumerate}
\begin{DoxyEnumerate}
\item Build {\ttfamily jasmine.\+js} with {\ttfamily grunt build\+Distribution} and run all specs again -\/ this ensures that your changes self-\/test well
\end{DoxyEnumerate}

\subsection*{Submitting a Pull Request}


\begin{DoxyEnumerate}
\item Revert your changes to {\ttfamily jasmine.\+js} and {\ttfamily jasmine-\/html.\+js}
\begin{DoxyItemize}
\item We do this because {\ttfamily jasmine.\+js} and {\ttfamily jasmine-\/html.\+js} are auto-\/generated (as you\textquotesingle{}ve seen in the previous steps) and accepting multiple pull requests when this auto-\/generated file changes causes lots of headaches
\end{DoxyItemize}
\end{DoxyEnumerate}
\begin{DoxyEnumerate}
\item When we accept your pull request, we will generate these files as a separate commit and merge the entire branch into master
\end{DoxyEnumerate}

Note that we use Travis for Continuous Integration. We only accept green pull requests. 