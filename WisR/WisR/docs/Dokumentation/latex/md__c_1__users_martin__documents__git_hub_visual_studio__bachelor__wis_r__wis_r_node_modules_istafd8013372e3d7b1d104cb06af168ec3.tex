Like the unix {\ttfamily which} utility.

Finds the first instance of a specified executable in the P\+A\+T\+H environment variable. Does not cache the results, so {\ttfamily hash -\/r} is not needed when the P\+A\+T\+H changes.

\subsection*{U\+S\+A\+G\+E}


\begin{DoxyCode}
1 var which = require('which')
2 
3 // async usage
4 which('node', function (er, resolvedPath) \{
5   // er is returned if no "node" is found on the PATH
6   // if it is found, then the absolute path to the exec is returned
7 \})
8 
9 // sync usage
10 // throws if not found
11 var resolved = which.sync('node')
12 
13 // Pass options to override the PATH and PATHEXT environment vars.
14 which('node', \{ path: someOtherPath \}, function (er, resolved) \{
15   if (er)
16     throw er
17   console.log('found at %j', resolved)
18 \})
\end{DoxyCode}


\subsection*{O\+P\+T\+I\+O\+N\+S}

If you pass in options, then {\ttfamily path} and {\ttfamily path\+Ext} are relevant. 