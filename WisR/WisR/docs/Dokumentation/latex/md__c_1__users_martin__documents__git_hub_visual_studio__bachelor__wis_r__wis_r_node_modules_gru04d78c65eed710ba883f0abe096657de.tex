\section*{~\newline
 ~\newline
  ~\newline
 ~\newline
 ~\newline
 }

\begin{quote}
Terminal string styling done right \end{quote}


\href{https://travis-ci.org/chalk/chalk}{\tt !\mbox{[}Build Status\mbox{]}(https\+://travis-\/ci.\+org/chalk/chalk.\+svg?branch=master)} \href{https://coveralls.io/r/chalk/chalk?branch=master}{\tt !\mbox{[}Coverage Status\mbox{]}(https\+://coveralls.\+io/repos/chalk/chalk/badge.\+svg?branch=master)} \href{https://www.youtube.com/watch?v=9auOCbH5Ns4}{\tt !\mbox{[}\mbox{]}(http\+://img.\+shields.\+io/badge/unicorn-\/approved-\/ff69b4.\+svg)}

\href{https://github.com/Marak/colors.js}{\tt colors.\+js} used to be the most popular string styling module, but it has serious deficiencies like extending {\ttfamily String.\+prototype} which causes all kinds of \href{https://github.com/yeoman/yo/issues/68}{\tt problems}. Although there are other ones, they either do too much or not enough.

{\bfseries Chalk is a clean and focused alternative.}



\subsection*{Why}


\begin{DoxyItemize}
\item Highly performant
\item Doesn\textquotesingle{}t extend {\ttfamily String.\+prototype}
\item Expressive A\+P\+I
\item Ability to nest styles
\item Clean and focused
\item Auto-\/detects color support
\item Actively maintained
\item \href{https://www.npmjs.com/browse/depended/chalk}{\tt Used by $\sim$4500 modules} as of July 15, 2015
\end{DoxyItemize}

\subsection*{Install}


\begin{DoxyCode}
1 $ npm install --save chalk
\end{DoxyCode}


\subsection*{Usage}

Chalk comes with an easy to use composable A\+P\+I where you just chain and nest the styles you want.


\begin{DoxyCode}
var chalk = require(\textcolor{stringliteral}{'chalk'});

\textcolor{comment}{// style a string}
chalk.blue(\textcolor{stringliteral}{'Hello world!'});

\textcolor{comment}{// combine styled and normal strings}
chalk.blue(\textcolor{stringliteral}{'Hello'}) + \textcolor{stringliteral}{'World'} + chalk.red(\textcolor{charliteral}{'!'});

\textcolor{comment}{// compose multiple styles using the chainable API}
chalk.blue.bgRed.bold(\textcolor{stringliteral}{'Hello world!'});

\textcolor{comment}{// pass in multiple arguments}
chalk.blue(\textcolor{stringliteral}{'Hello'}, \textcolor{stringliteral}{'World!'}, \textcolor{stringliteral}{'Foo'}, \textcolor{stringliteral}{'bar'}, \textcolor{stringliteral}{'biz'}, \textcolor{stringliteral}{'baz'});

\textcolor{comment}{// nest styles}
chalk.red(\textcolor{stringliteral}{'Hello'}, chalk.underline.bgBlue(\textcolor{stringliteral}{'world'}) + \textcolor{charliteral}{'!'});

\textcolor{comment}{// nest styles of the same type even (color, underline, background)}
chalk.green(
    \textcolor{stringliteral}{'I am a green line '} +
    chalk.blue.underline.bold(\textcolor{stringliteral}{'with a blue substring'}) +
    \textcolor{stringliteral}{' that becomes green again!'}
);
\end{DoxyCode}


Easily define your own themes.


\begin{DoxyCode}
var chalk = require(\textcolor{stringliteral}{'chalk'});
var error = chalk.bold.red;
console.log(error(\textcolor{stringliteral}{'Error!'}));
\end{DoxyCode}


Take advantage of console.\+log \href{http://nodejs.org/docs/latest/api/console.html#console_console_log_data}{\tt string substitution}.


\begin{DoxyCode}
var name = \textcolor{stringliteral}{'Sindre'};
console.log(chalk.green(\textcolor{stringliteral}{'Hello %s'}), name);
\textcolor{comment}{//=> Hello Sindre}
\end{DoxyCode}


\subsection*{A\+P\+I}

\subsubsection*{chalk.{\ttfamily $<$style$>$\mbox{[}.$<$style$>$...\mbox{]}(string, \mbox{[}string...\mbox{]})}}

Example\+: `chalk.red.\+bold.\+underline(\textquotesingle{}Hello\textquotesingle{}, \textquotesingle{}world\textquotesingle{});`

Chain \href{#styles}{\tt styles} and call the last one as a method with a string argument. Order doesn\textquotesingle{}t matter, and later styles take precedent in case of a conflict. This simply means that {\ttfamily Chalk.\+red.\+yellow.\+green} is equivalent to {\ttfamily Chalk.\+green}.

Multiple arguments will be separated by space.

\subsubsection*{chalk.\+enabled}

Color support is automatically detected, but you can override it by setting the {\ttfamily enabled} property. You should however only do this in your own code as it applies globally to all chalk consumers.

If you need to change this in a reusable module create a new instance\+:


\begin{DoxyCode}
var ctx = \textcolor{keyword}{new} chalk.constructor(\{enabled: \textcolor{keyword}{false}\});
\end{DoxyCode}


\subsubsection*{chalk.\+supports\+Color}

Detect whether the terminal \href{https://github.com/chalk/supports-color}{\tt supports color}. Used internally and handled for you, but exposed for convenience.

Can be overridden by the user with the flags {\ttfamily -\/-\/color} and {\ttfamily -\/-\/no-\/color}. For situations where using {\ttfamily -\/-\/color} is not possible, add an environment variable {\ttfamily F\+O\+R\+C\+E\+\_\+\+C\+O\+L\+O\+R} with any value to force color. Trumps {\ttfamily -\/-\/no-\/color}.

\subsubsection*{chalk.\+styles}

Exposes the styles as \href{https://github.com/chalk/ansi-styles}{\tt A\+N\+S\+I escape codes}.

Generally not useful, but you might need just the {\ttfamily .open} or {\ttfamily .close} escape code if you\textquotesingle{}re mixing externally styled strings with your own.


\begin{DoxyCode}
var chalk = require(\textcolor{stringliteral}{'chalk'});

console.log(chalk.styles.red);
\textcolor{comment}{//=> \{open: '\(\backslash\)u001b[31m', close: '\(\backslash\)u001b[39m'\}}

console.log(chalk.styles.red.open + \textcolor{stringliteral}{'Hello'} + chalk.styles.red.close);
\end{DoxyCode}


\subsubsection*{chalk.\+has\+Color(string)}

Check whether a string \href{https://github.com/chalk/has-ansi}{\tt has color}.

\subsubsection*{chalk.\+strip\+Color(string)}

\href{https://github.com/chalk/strip-ansi}{\tt Strip color} from a string.

Can be useful in combination with {\ttfamily .supports\+Color} to strip color on externally styled text when it\textquotesingle{}s not supported.

Example\+:


\begin{DoxyCode}
var chalk = require(\textcolor{stringliteral}{'chalk'});
var styledString = getText();

\textcolor{keywordflow}{if} (!chalk.supportsColor) \{
    styledString = chalk.stripColor(styledString);
\}
\end{DoxyCode}


\subsection*{Styles}

\subsubsection*{Modifiers}


\begin{DoxyItemize}
\item {\ttfamily reset}
\item {\ttfamily bold}
\item {\ttfamily dim}
\item {\ttfamily italic} $\ast$(not widely supported)$\ast$
\item {\ttfamily underline}
\item {\ttfamily inverse}
\item {\ttfamily hidden}
\item {\ttfamily strikethrough} $\ast$(not widely supported)$\ast$
\end{DoxyItemize}

\subsubsection*{Colors}


\begin{DoxyItemize}
\item {\ttfamily black}
\item {\ttfamily red}
\item {\ttfamily green}
\item {\ttfamily yellow}
\item {\ttfamily blue} $\ast$(on Windows the bright version is used as normal blue is illegible)$\ast$
\item {\ttfamily magenta}
\item {\ttfamily cyan}
\item {\ttfamily white}
\item {\ttfamily gray}
\end{DoxyItemize}

\subsubsection*{Background colors}


\begin{DoxyItemize}
\item {\ttfamily bg\+Black}
\item {\ttfamily bg\+Red}
\item {\ttfamily bg\+Green}
\item {\ttfamily bg\+Yellow}
\item {\ttfamily bg\+Blue}
\item {\ttfamily bg\+Magenta}
\item {\ttfamily bg\+Cyan}
\item {\ttfamily bg\+White}
\end{DoxyItemize}

\subsection*{256-\/colors}

Chalk does not support anything other than the base eight colors, which guarantees it will work on all terminals and systems. Some terminals, specifically {\ttfamily xterm} compliant ones, will support the full range of 8-\/bit colors. For this the lower level \href{https://github.com/jbnicolai/ansi-256-colors}{\tt ansi-\/256-\/colors} package can be used.

\subsection*{Windows}

If you\textquotesingle{}re on Windows, do yourself a favor and use \href{http://bliker.github.io/cmder/}{\tt `cmder`} instead of {\ttfamily cmd.\+exe}.

\subsection*{Related}


\begin{DoxyItemize}
\item \href{https://github.com/chalk/chalk-cli}{\tt chalk-\/cli} -\/ C\+L\+I for this module
\item \href{https://github.com/chalk/ansi-styles/}{\tt ansi-\/styles} -\/ A\+N\+S\+I escape codes for styling strings in the terminal
\item \href{https://github.com/chalk/supports-color/}{\tt supports-\/color} -\/ Detect whether a terminal supports color
\item \href{https://github.com/chalk/strip-ansi}{\tt strip-\/ansi} -\/ Strip A\+N\+S\+I escape codes
\item \href{https://github.com/chalk/has-ansi}{\tt has-\/ansi} -\/ Check if a string has A\+N\+S\+I escape codes
\item \href{https://github.com/chalk/ansi-regex}{\tt ansi-\/regex} -\/ Regular expression for matching A\+N\+S\+I escape codes
\item \href{https://github.com/chalk/wrap-ansi}{\tt wrap-\/ansi} -\/ Wordwrap a string with A\+N\+S\+I escape codes
\end{DoxyItemize}

\subsection*{License}

M\+I\+T © \href{http://sindresorhus.com}{\tt Sindre Sorhus} 