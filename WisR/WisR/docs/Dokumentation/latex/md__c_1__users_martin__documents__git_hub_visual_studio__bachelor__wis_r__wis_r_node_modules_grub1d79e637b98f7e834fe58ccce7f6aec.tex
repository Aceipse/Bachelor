\begin{quote}
Grunt and Phantom\+J\+S, sitting in a tree. \end{quote}


\subsection*{Usage}

The best way to understand how this lib should be used is by looking at the \href{https://github.com/gruntjs/grunt-contrib-qunit}{\tt grunt-\/contrib-\/qunit} plugin. Mainly, look at how \href{https://github.com/gruntjs/grunt-contrib-qunit/blob/d99291713d32f84e50303d6e51eb2dab40b1deb6/tasks/qunit.js#L17}{\tt the lib is required}, how \href{https://github.com/gruntjs/grunt-contrib-qunit/blob/d99291713d32f84e50303d6e51eb2dab40b1deb6/tasks/qunit.js#L61-L144}{\tt event handlers are bound} and how \href{https://github.com/gruntjs/grunt-contrib-qunit/blob/d99291713d32f84e50303d6e51eb2dab40b1deb6/tasks/qunit.js#L177-L190}{\tt Phantom\+J\+S is actually spawned}.

Also, in the case of the grunt-\/contrib-\/qunit plugin, it\textquotesingle{}s important to know that the page being loaded into Phantom\+J\+S {\itshape doesn\textquotesingle{}t} know it will be loaded into Phantom\+J\+S, and as such doesn\textquotesingle{}t have any Phantom\+J\+S-\/$>$Grunt code in it. That communication code, aka. the \href{https://github.com/gruntjs/grunt-contrib-qunit/blob/d99291713d32f84e50303d6e51eb2dab40b1deb6/phantomjs/bridge.js}{\tt \char`\"{}bridge\char`\"{}}, is dynamically \href{https://github.com/gruntjs/grunt-contrib-qunit/blob/d99291713d32f84e50303d6e51eb2dab40b1deb6/tasks/qunit.js#L152}{\tt injected into the html page}.

\subsubsection*{Options}


\begin{DoxyItemize}
\item {\ttfamily timeout}\+: Phantom\+J\+S\textquotesingle{} timeout, in milliseconds.
\item {\ttfamily inject}\+: Java\+Script to inject into the page.
\item {\ttfamily page}\+: an object of options for the Phantom\+J\+S \href{https://github.com/ariya/phantomjs/wiki/API-Reference-WebPage}{\tt `page` object}.
\item {\ttfamily screenshot}\+: saves a screenshot on failure
\end{DoxyItemize}

\subsection*{An inline example}

If a Grunt task looked something like this\+:


\begin{DoxyCode}
grunt.registerTask(\textcolor{stringliteral}{'mytask'}, \textcolor{stringliteral}{'Integrate with phantomjs.'}, \textcolor{keyword}{function}() \{
  var phantomjs = require(\textcolor{stringliteral}{'grunt-lib-phantomjs'}).init(grunt);
  var errorCount = 0;

  \textcolor{comment}{// Handle any number of namespaced events like so.}
  phantomjs.on(\textcolor{stringliteral}{'mytask.ok'}, \textcolor{keyword}{function}(msg) \{
    grunt.log.writeln(msg);
  \});

  phantomjs.on(\textcolor{stringliteral}{'mytask.error'}, \textcolor{keyword}{function}(msg) \{
    errorCount++;
    grunt.log.error(msg);
  \});

  \textcolor{comment}{// Create some kind of "all done" event.}
  phantomjs.on(\textcolor{stringliteral}{'mytask.done'}, \textcolor{keyword}{function}() \{
    phantomjs.halt();
  \});

  \textcolor{comment}{// Built-in error handlers.}
  phantomjs.on(\textcolor{stringliteral}{'fail.load'}, \textcolor{keyword}{function}(url) \{
    phantomjs.halt();
    grunt.warn(\textcolor{stringliteral}{'PhantomJS unable to load URL.'});
  \});

  phantomjs.on(\textcolor{stringliteral}{'fail.timeout'}, \textcolor{keyword}{function}() \{
    phantomjs.halt();
    grunt.warn(\textcolor{stringliteral}{'PhantomJS timed out.'});
  \});

  \textcolor{comment}{// This task is async.}
  var done = this.async();

  \textcolor{comment}{// Spawn phantomjs}
  phantomjs.spawn(\textcolor{stringliteral}{'test.html'}, \{
    \textcolor{comment}{// Additional PhantomJS options.}
    options: \{\},
    \textcolor{comment}{// Complete the task when done.}
    done: \textcolor{keyword}{function}(err) \{
      done(err || errorCount === 0);
    \}
  \});

\});
\end{DoxyCode}


And {\ttfamily test.\+html} looked something like this (note the \char`\"{}bridge\char`\"{} is hard-\/coded into this page and not injected)\+:


\begin{DoxyCode}
1 <!doctype html>
2 <html>
3 <head>
4 <script>
5 
6 // Send messages to the parent PhantomJS process via alert! Good times!!
7 function sendMessage() \{
8   var args = [].slice.call(arguments);
9   alert(JSON.stringify(args));
10 \}
11 
12 sendMessage('mytask.ok', 'Something worked.');
13 sendMessage('mytask.error', 'Something failed.');
14 sendMessage('mytask.done');
15 
16 </script>
17 </head>
18 <body>
19 </body>
20 </html>
\end{DoxyCode}


Then running Grunt would behave something like this\+:


\begin{DoxyCode}
1 $ grunt mytask
2 Running "mytask" task
3 Something worked.
4 >> Something failed.
5 Warning: Task "mytask" failed. Use --force to continue.
6 
7 Aborted due to warnings.
\end{DoxyCode}


\subsection*{O\+S Dependencies}

Phantom\+J\+S requires these dependencies on Ubuntu/\+Debian\+:


\begin{DoxyCode}
1 apt-get install libfontconfig1 fontconfig libfontconfig1-dev libfreetype6-dev
\end{DoxyCode}
 