Only call a function once.

\subsection*{usage}


\begin{DoxyCode}
1 var once = require('once')
2 
3 function load (file, cb) \{
4   cb = once(cb)
5   loader.load('file')
6   loader.once('load', cb)
7   loader.once('error', cb)
8 \}
\end{DoxyCode}


Or add to the Function.\+prototype in a responsible way\+:


\begin{DoxyCode}
1 // only has to be done once
2 require('once').proto()
3 
4 function load (file, cb) \{
5   cb = cb.once()
6   loader.load('file')
7   loader.once('load', cb)
8   loader.once('error', cb)
9 \}
\end{DoxyCode}


Ironically, the prototype feature makes this module twice as complicated as necessary.

To check whether you function has been called, use {\ttfamily fn.\+called}. Once the function is called for the first time the return value of the original function is saved in {\ttfamily fn.\+value} and subsequent calls will continue to return this value.


\begin{DoxyCode}
1 var once = require('once')
2 
3 function load (cb) \{
4   cb = once(cb)
5   var stream = createStream()
6   stream.once('data', cb)
7   stream.once('end', function () \{
8     if (!cb.called) cb(new Error('not found'))
9   \})
10 \}
\end{DoxyCode}
 