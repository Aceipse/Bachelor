Async is a utility module which provides straight-\/forward, powerful functions for working with asynchronous Java\+Script. Although originally designed for use with \href{http://nodejs.org}{\tt node.\+js}, it can also be used directly in the browser.

Async provides around 20 functions that include the usual \textquotesingle{}functional\textquotesingle{} suspects (map, reduce, filter, for\+Each…) as well as some common patterns for asynchronous control flow (parallel, series, waterfall…). All these functions assume you follow the node.\+js convention of providing a single callback as the last argument of your async function.

\subsection*{Quick Examples}

\begin{DoxyVerb}async.map(['file1','file2','file3'], fs.stat, function(err, results){
    // results is now an array of stats for each file
});

async.filter(['file1','file2','file3'], path.exists, function(results){
    // results now equals an array of the existing files
});

async.parallel([
    function(){ ... },
    function(){ ... }
], callback);

async.series([
    function(){ ... },
    function(){ ... }
]);
\end{DoxyVerb}


There are many more functions available so take a look at the docs below for a full list. This module aims to be comprehensive, so if you feel anything is missing please create a Git\+Hub issue for it.

\subsection*{Download}

Releases are available for download from \href{http://github.com/caolan/async/downloads}{\tt Git\+Hub}. Alternatively, you can install using Node Package Manager (npm)\+: \begin{DoxyVerb}npm install async
\end{DoxyVerb}


{\bfseries Development\+:} \href{https://github.com/caolan/async/raw/master/lib/async.js}{\tt async.\+js} -\/ 17.\+5kb Uncompressed

{\bfseries Production\+:} \href{https://github.com/caolan/async/raw/master/dist/async.min.js}{\tt async.\+min.\+js} -\/ 1.\+7kb Packed and Gzipped

\subsection*{In the Browser}

So far its been tested in I\+E6, I\+E7, I\+E8, F\+F3.\+6 and Chrome 5. Usage\+: \begin{DoxyVerb}<script type="text/javascript" src="async.js"></script>
<script type="text/javascript">

    async.map(data, asyncProcess, function(err, results){
        alert(results);
    });

</script>
\end{DoxyVerb}


\subsection*{Documentation}

\subsubsection*{Collections}


\begin{DoxyItemize}
\item \href{#forEach}{\tt for\+Each}
\item \href{#map}{\tt map}
\item \href{#filter}{\tt filter}
\item \href{#reject}{\tt reject}
\item \href{#reduce}{\tt reduce}
\item \href{#detect}{\tt detect}
\item \href{#sortBy}{\tt sort\+By}
\item \href{#some}{\tt some}
\item \href{#every}{\tt every}
\item \href{#concat}{\tt concat}
\end{DoxyItemize}

\subsubsection*{Control Flow}


\begin{DoxyItemize}
\item \href{#series}{\tt series}
\item \href{#parallel}{\tt parallel}
\item \href{#whilst}{\tt whilst}
\item \href{#until}{\tt until}
\item \href{#waterfall}{\tt waterfall}
\item \href{#queue}{\tt queue}
\item \href{#auto}{\tt auto}
\item \href{#iterator}{\tt iterator}
\item \href{#apply}{\tt apply}
\item \href{#nextTick}{\tt next\+Tick}
\end{DoxyItemize}

\subsubsection*{Utils}


\begin{DoxyItemize}
\item \href{#memoize}{\tt memoize}
\item \href{#unmemoize}{\tt unmemoize}
\item \href{#log}{\tt log}
\item \href{#dir}{\tt dir}
\item \href{#noConflict}{\tt no\+Conflict}
\end{DoxyItemize}

\subsection*{Collections}

\label{_forEach}%
 \subsubsection*{for\+Each(arr, iterator, callback)}

Applies an iterator function to each item in an array, in parallel. The iterator is called with an item from the list and a callback for when it has finished. If the iterator passes an error to this callback, the main callback for the for\+Each function is immediately called with the error.

Note, that since this function applies the iterator to each item in parallel there is no guarantee that the iterator functions will complete in order.

{\bfseries Arguments}


\begin{DoxyItemize}
\item arr -\/ An array to iterate over.
\item iterator(item, callback) -\/ A function to apply to each item in the array. The iterator is passed a callback which must be called once it has completed.
\item callback(err) -\/ A callback which is called after all the iterator functions have finished, or an error has occurred.
\end{DoxyItemize}

{\bfseries Example} \begin{DoxyVerb}// assuming openFiles is an array of file names and saveFile is a function
// to save the modified contents of that file:

async.forEach(openFiles, saveFile, function(err){
    // if any of the saves produced an error, err would equal that error
});
\end{DoxyVerb}






\label{_forEachSeries}%
 \subsubsection*{for\+Each\+Series(arr, iterator, callback)}

The same as for\+Each only the iterator is applied to each item in the array in series. The next iterator is only called once the current one has completed processing. This means the iterator functions will complete in order.





\label{_forEachLimit}%
 \subsubsection*{for\+Each\+Limit(arr, limit, iterator, callback)}

The same as for\+Each only the iterator is applied to batches of items in the array, in series. The next batch of iterators is only called once the current one has completed processing.

{\bfseries Arguments}


\begin{DoxyItemize}
\item arr -\/ An array to iterate over.
\item limit -\/ How many items should be in each batch.
\item iterator(item, callback) -\/ A function to apply to each item in the array. The iterator is passed a callback which must be called once it has completed.
\item callback(err) -\/ A callback which is called after all the iterator functions have finished, or an error has occurred.
\end{DoxyItemize}

{\bfseries Example} \begin{DoxyVerb}// Assume documents is an array of JSON objects and requestApi is a
// function that interacts with a rate-limited REST api.

async.forEachLimit(documents, 20, requestApi, function(err){
    // if any of the saves produced an error, err would equal that error
});
\end{DoxyVerb}
 



\label{_map}%
 \subsubsection*{map(arr, iterator, callback)}

Produces a new array of values by mapping each value in the given array through the iterator function. The iterator is called with an item from the array and a callback for when it has finished processing. The callback takes 2 arguments, an error and the transformed item from the array. If the iterator passes an error to this callback, the main callback for the map function is immediately called with the error.

Note, that since this function applies the iterator to each item in parallel there is no guarantee that the iterator functions will complete in order, however the results array will be in the same order as the original array.

{\bfseries Arguments}


\begin{DoxyItemize}
\item arr -\/ An array to iterate over.
\item iterator(item, callback) -\/ A function to apply to each item in the array. The iterator is passed a callback which must be called once it has completed with an error (which can be null) and a transformed item.
\item callback(err, results) -\/ A callback which is called after all the iterator functions have finished, or an error has occurred. Results is an array of the transformed items from the original array.
\end{DoxyItemize}

{\bfseries Example} \begin{DoxyVerb}async.map(['file1','file2','file3'], fs.stat, function(err, results){
    // results is now an array of stats for each file
});
\end{DoxyVerb}






\label{_mapSeries}%
 \subsubsection*{map\+Series(arr, iterator, callback)}

The same as map only the iterator is applied to each item in the array in series. The next iterator is only called once the current one has completed processing. The results array will be in the same order as the original.





\label{_filter}%
 \subsubsection*{filter(arr, iterator, callback)}

{\bfseries Alias\+:} select

Returns a new array of all the values which pass an async truth test. {\itshape The callback for each iterator call only accepts a single argument of true or false, it does not accept an error argument first!} This is in-\/line with the way node libraries work with truth tests like path.\+exists. This operation is performed in parallel, but the results array will be in the same order as the original.

{\bfseries Arguments}


\begin{DoxyItemize}
\item arr -\/ An array to iterate over.
\item iterator(item, callback) -\/ A truth test to apply to each item in the array. The iterator is passed a callback which must be called once it has completed.
\item callback(results) -\/ A callback which is called after all the iterator functions have finished.
\end{DoxyItemize}

{\bfseries Example} \begin{DoxyVerb}async.filter(['file1','file2','file3'], path.exists, function(results){
    // results now equals an array of the existing files
});
\end{DoxyVerb}






\label{_filterSeries}%
 \subsubsection*{filter\+Series(arr, iterator, callback)}

{\bfseries alias\+:} select\+Series

The same as filter only the iterator is applied to each item in the array in series. The next iterator is only called once the current one has completed processing. The results array will be in the same order as the original. 



\label{_reject}%
 \subsubsection*{reject(arr, iterator, callback)}

The opposite of filter. Removes values that pass an async truth test. 



\label{_rejectSeries}%
 \subsubsection*{reject\+Series(arr, iterator, callback)}

The same as filter, only the iterator is applied to each item in the array in series.





\label{_reduce}%
 \subsubsection*{reduce(arr, memo, iterator, callback)}

{\bfseries aliases\+:} inject, foldl

Reduces a list of values into a single value using an async iterator to return each successive step. Memo is the initial state of the reduction. This function only operates in series. For performance reasons, it may make sense to split a call to this function into a parallel map, then use the normal Array.\+prototype.\+reduce on the results. This function is for situations where each step in the reduction needs to be async, if you can get the data before reducing it then its probably a good idea to do so.

{\bfseries Arguments}


\begin{DoxyItemize}
\item arr -\/ An array to iterate over.
\item memo -\/ The initial state of the reduction.
\item iterator(memo, item, callback) -\/ A function applied to each item in the array to produce the next step in the reduction. The iterator is passed a callback which accepts an optional error as its first argument, and the state of the reduction as the second. If an error is passed to the callback, the reduction is stopped and the main callback is immediately called with the error.
\item callback(err, result) -\/ A callback which is called after all the iterator functions have finished. Result is the reduced value.
\end{DoxyItemize}

{\bfseries Example} \begin{DoxyVerb}async.reduce([1,2,3], 0, function(memo, item, callback){
    // pointless async:
    process.nextTick(function(){
        callback(null, memo + item)
    });
}, function(err, result){
    // result is now equal to the last value of memo, which is 6
});
\end{DoxyVerb}






\label{_reduceRight}%
 \subsubsection*{reduce\+Right(arr, memo, iterator, callback)}

{\bfseries Alias\+:} foldr

Same as reduce, only operates on the items in the array in reverse order.





\label{_detect}%
 \subsubsection*{detect(arr, iterator, callback)}

Returns the first value in a list that passes an async truth test. The iterator is applied in parallel, meaning the first iterator to return true will fire the detect callback with that result. That means the result might not be the first item in the original array (in terms of order) that passes the test.

If order within the original array is important then look at detect\+Series.

{\bfseries Arguments}


\begin{DoxyItemize}
\item arr -\/ An array to iterate over.
\item iterator(item, callback) -\/ A truth test to apply to each item in the array. The iterator is passed a callback which must be called once it has completed.
\item callback(result) -\/ A callback which is called as soon as any iterator returns true, or after all the iterator functions have finished. Result will be the first item in the array that passes the truth test (iterator) or the value undefined if none passed.
\end{DoxyItemize}

{\bfseries Example} \begin{DoxyVerb}async.detect(['file1','file2','file3'], path.exists, function(result){
    // result now equals the first file in the list that exists
});
\end{DoxyVerb}






\label{_detectSeries}%
 \subsubsection*{detect\+Series(arr, iterator, callback)}

The same as detect, only the iterator is applied to each item in the array in series. This means the result is always the first in the original array (in terms of array order) that passes the truth test.





\label{_sortBy}%
 \subsubsection*{sort\+By(arr, iterator, callback)}

Sorts a list by the results of running each value through an async iterator.

{\bfseries Arguments}


\begin{DoxyItemize}
\item arr -\/ An array to iterate over.
\item iterator(item, callback) -\/ A function to apply to each item in the array. The iterator is passed a callback which must be called once it has completed with an error (which can be null) and a value to use as the sort criteria.
\item callback(err, results) -\/ A callback which is called after all the iterator functions have finished, or an error has occurred. Results is the items from the original array sorted by the values returned by the iterator calls.
\end{DoxyItemize}

{\bfseries Example} \begin{DoxyVerb}async.sortBy(['file1','file2','file3'], function(file, callback){
    fs.stat(file, function(err, stats){
        callback(err, stats.mtime);
    });
}, function(err, results){
    // results is now the original array of files sorted by
    // modified date
});
\end{DoxyVerb}






\label{_some}%
 \subsubsection*{some(arr, iterator, callback)}

{\bfseries Alias\+:} any

Returns true if at least one element in the array satisfies an async test. {\itshape The callback for each iterator call only accepts a single argument of true or false, it does not accept an error argument first!} This is in-\/line with the way node libraries work with truth tests like path.\+exists. Once any iterator call returns true, the main callback is immediately called.

{\bfseries Arguments}


\begin{DoxyItemize}
\item arr -\/ An array to iterate over.
\item iterator(item, callback) -\/ A truth test to apply to each item in the array. The iterator is passed a callback which must be called once it has completed.
\item callback(result) -\/ A callback which is called as soon as any iterator returns true, or after all the iterator functions have finished. Result will be either true or false depending on the values of the async tests.
\end{DoxyItemize}

{\bfseries Example} \begin{DoxyVerb}async.some(['file1','file2','file3'], path.exists, function(result){
    // if result is true then at least one of the files exists
});
\end{DoxyVerb}






\label{_every}%
 \subsubsection*{every(arr, iterator, callback)}

{\bfseries Alias\+:} all

Returns true if every element in the array satisfies an async test. {\itshape The callback for each iterator call only accepts a single argument of true or false, it does not accept an error argument first!} This is in-\/line with the way node libraries work with truth tests like path.\+exists.

{\bfseries Arguments}


\begin{DoxyItemize}
\item arr -\/ An array to iterate over.
\item iterator(item, callback) -\/ A truth test to apply to each item in the array. The iterator is passed a callback which must be called once it has completed.
\item callback(result) -\/ A callback which is called after all the iterator functions have finished. Result will be either true or false depending on the values of the async tests.
\end{DoxyItemize}

{\bfseries Example} \begin{DoxyVerb}async.every(['file1','file2','file3'], path.exists, function(result){
    // if result is true then every file exists
});
\end{DoxyVerb}






\label{_concat}%
 \subsubsection*{concat(arr, iterator, callback)}

Applies an iterator to each item in a list, concatenating the results. Returns the concatenated list. The iterators are called in parallel, and the results are concatenated as they return. There is no guarantee that the results array will be returned in the original order of the arguments passed to the iterator function.

{\bfseries Arguments}


\begin{DoxyItemize}
\item arr -\/ An array to iterate over
\item iterator(item, callback) -\/ A function to apply to each item in the array. The iterator is passed a callback which must be called once it has completed with an error (which can be null) and an array of results.
\item callback(err, results) -\/ A callback which is called after all the iterator functions have finished, or an error has occurred. Results is an array containing the concatenated results of the iterator function.
\end{DoxyItemize}

{\bfseries Example} \begin{DoxyVerb}async.concat(['dir1','dir2','dir3'], fs.readdir, function(err, files){
    // files is now a list of filenames that exist in the 3 directories
});
\end{DoxyVerb}






\label{_concatSeries}%
 \subsubsection*{concat\+Series(arr, iterator, callback)}

Same as async.\+concat, but executes in series instead of parallel.

\subsection*{Control Flow}

\label{_series}%
 \subsubsection*{series(tasks, \mbox{[}callback\mbox{]})}

Run an array of functions in series, each one running once the previous function has completed. If any functions in the series pass an error to its callback, no more functions are run and the callback for the series is immediately called with the value of the error. Once the tasks have completed, the results are passed to the final callback as an array.

It is also possible to use an object instead of an array. Each property will be run as a function and the results will be passed to the final callback as an object instead of an array. This can be a more readable way of handling results from async.\+series.

{\bfseries Arguments}


\begin{DoxyItemize}
\item tasks -\/ An array or object containing functions to run, each function is passed a callback it must call on completion.
\item callback(err, results) -\/ An optional callback to run once all the functions have completed. This function gets an array of all the arguments passed to the callbacks used in the array.
\end{DoxyItemize}

{\bfseries Example} \begin{DoxyVerb}async.series([
    function(callback){
        // do some stuff ...
        callback(null, 'one');
    },
    function(callback){
        // do some more stuff ...
        callback(null, 'two');
    },
],
// optional callback
function(err, results){
    // results is now equal to ['one', 'two']
});


// an example using an object instead of an array
async.series({
    one: function(callback){
        setTimeout(function(){
            callback(null, 1);
        }, 200);
    },
    two: function(callback){
        setTimeout(function(){
            callback(null, 2);
        }, 100);
    },
},
function(err, results) {
    // results is now equal to: {one: 1, two: 2}
});
\end{DoxyVerb}






\label{_parallel}%
 \subsubsection*{parallel(tasks, \mbox{[}callback\mbox{]})}

Run an array of functions in parallel, without waiting until the previous function has completed. If any of the functions pass an error to its callback, the main callback is immediately called with the value of the error. Once the tasks have completed, the results are passed to the final callback as an array.

It is also possible to use an object instead of an array. Each property will be run as a function and the results will be passed to the final callback as an object instead of an array. This can be a more readable way of handling results from async.\+parallel.

{\bfseries Arguments}


\begin{DoxyItemize}
\item tasks -\/ An array or object containing functions to run, each function is passed a callback it must call on completion.
\item callback(err, results) -\/ An optional callback to run once all the functions have completed. This function gets an array of all the arguments passed to the callbacks used in the array.
\end{DoxyItemize}

{\bfseries Example} \begin{DoxyVerb}async.parallel([
    function(callback){
        setTimeout(function(){
            callback(null, 'one');
        }, 200);
    },
    function(callback){
        setTimeout(function(){
            callback(null, 'two');
        }, 100);
    },
],
// optional callback
function(err, results){
    // the results array will equal ['one','two'] even though
    // the second function had a shorter timeout.
});


// an example using an object instead of an array
async.parallel({
    one: function(callback){
        setTimeout(function(){
            callback(null, 1);
        }, 200);
    },
    two: function(callback){
        setTimeout(function(){
            callback(null, 2);
        }, 100);
    },
},
function(err, results) {
    // results is now equals to: {one: 1, two: 2}
});
\end{DoxyVerb}






\label{_whilst}%
 \subsubsection*{whilst(test, fn, callback)}

Repeatedly call fn, while test returns true. Calls the callback when stopped, or an error occurs.

{\bfseries Arguments}


\begin{DoxyItemize}
\item test() -\/ synchronous truth test to perform before each execution of fn.
\item fn(callback) -\/ A function to call each time the test passes. The function is passed a callback which must be called once it has completed with an optional error as the first argument.
\item callback(err) -\/ A callback which is called after the test fails and repeated execution of fn has stopped.
\end{DoxyItemize}

{\bfseries Example} \begin{DoxyVerb}var count = 0;

async.whilst(
    function () { return count < 5; },
    function (callback) {
        count++;
        setTimeout(callback, 1000);
    },
    function (err) {
        // 5 seconds have passed
    }
);
\end{DoxyVerb}






\label{_until}%
 \subsubsection*{until(test, fn, callback)}

Repeatedly call fn, until test returns true. Calls the callback when stopped, or an error occurs.

The inverse of async.\+whilst.





\label{_waterfall}%
 \subsubsection*{waterfall(tasks, \mbox{[}callback\mbox{]})}

Runs an array of functions in series, each passing their results to the next in the array. However, if any of the functions pass an error to the callback, the next function is not executed and the main callback is immediately called with the error.

{\bfseries Arguments}


\begin{DoxyItemize}
\item tasks -\/ An array of functions to run, each function is passed a callback it must call on completion.
\item callback(err, \mbox{[}results\mbox{]}) -\/ An optional callback to run once all the functions have completed. This will be passed the results of the last task\textquotesingle{}s callback.
\end{DoxyItemize}

{\bfseries Example} \begin{DoxyVerb}async.waterfall([
    function(callback){
        callback(null, 'one', 'two');
    },
    function(arg1, arg2, callback){
        callback(null, 'three');
    },
    function(arg1, callback){
        // arg1 now equals 'three'
        callback(null, 'done');
    }
], function (err, result) {
   // result now equals 'done'    
});
\end{DoxyVerb}






\label{_queue}%
 \subsubsection*{queue(worker, concurrency)}

Creates a queue object with the specified concurrency. Tasks added to the queue will be processed in parallel (up to the concurrency limit). If all workers are in progress, the task is queued until one is available. Once a worker has completed a task, the task\textquotesingle{}s callback is called.

{\bfseries Arguments}


\begin{DoxyItemize}
\item worker(task, callback) -\/ An asynchronous function for processing a queued task.
\item concurrency -\/ An integer for determining how many worker functions should be run in parallel.
\end{DoxyItemize}

{\bfseries Queue objects}

The queue object returned by this function has the following properties and methods\+:


\begin{DoxyItemize}
\item length() -\/ a function returning the number of items waiting to be processed.
\item concurrency -\/ an integer for determining how many worker functions should be run in parallel. This property can be changed after a queue is created to alter the concurrency on-\/the-\/fly.
\item push(task, \mbox{[}callback\mbox{]}) -\/ add a new task to the queue, the callback is called once the worker has finished processing the task. instead of a single task, an array of tasks can be submitted. the respective callback is used for every task in the list.
\item saturated -\/ a callback that is called when the queue length hits the concurrency and further tasks will be queued
\item empty -\/ a callback that is called when the last item from the queue is given to a worker
\item drain -\/ a callback that is called when the last item from the queue has returned from the worker
\end{DoxyItemize}

{\bfseries Example} \begin{DoxyVerb}// create a queue object with concurrency 2

var q = async.queue(function (task, callback) {
    console.log('hello ' + task.name);
    callback();
}, 2);


// assign a callback
q.drain = function() {
    console.log('all items have been processed');
}

// add some items to the queue

q.push({name: 'foo'}, function (err) {
    console.log('finished processing foo');
});
q.push({name: 'bar'}, function (err) {
    console.log('finished processing bar');
});

// add some items to the queue (batch-wise)

q.push([{name: 'baz'},{name: 'bay'},{name: 'bax'}], function (err) {
    console.log('finished processing bar');
});
\end{DoxyVerb}






\label{_auto}%
 \subsubsection*{auto(tasks, \mbox{[}callback\mbox{]})}

Determines the best order for running functions based on their requirements. Each function can optionally depend on other functions being completed first, and each function is run as soon as its requirements are satisfied. If any of the functions pass an error to their callback, that function will not complete (so any other functions depending on it will not run) and the main callback will be called immediately with the error. Functions also receive an object containing the results of functions which have completed so far.

{\bfseries Arguments}


\begin{DoxyItemize}
\item tasks -\/ An object literal containing named functions or an array of requirements, with the function itself the last item in the array. The key used for each function or array is used when specifying requirements. The syntax is easier to understand by looking at the example.
\item callback(err, results) -\/ An optional callback which is called when all the tasks have been completed. The callback will receive an error as an argument if any tasks pass an error to their callback. If all tasks complete successfully, it will receive an object containing their results.
\end{DoxyItemize}

{\bfseries Example} \begin{DoxyVerb}async.auto({
    get_data: function(callback){
        // async code to get some data
    },
    make_folder: function(callback){
        // async code to create a directory to store a file in
        // this is run at the same time as getting the data
    },
    write_file: ['get_data', 'make_folder', function(callback){
        // once there is some data and the directory exists,
        // write the data to a file in the directory
        callback(null, filename);
    }],
    email_link: ['write_file', function(callback, results){
        // once the file is written let's email a link to it...
        // results.write_file contains the filename returned by write_file.
    }]
});
\end{DoxyVerb}


This is a fairly trivial example, but to do this using the basic parallel and series functions would look like this\+: \begin{DoxyVerb}async.parallel([
    function(callback){
        // async code to get some data
    },
    function(callback){
        // async code to create a directory to store a file in
        // this is run at the same time as getting the data
    }
],
function(results){
    async.series([
        function(callback){
            // once there is some data and the directory exists,
            // write the data to a file in the directory
        },
        email_link: function(callback){
            // once the file is written let's email a link to it...
        }
    ]);
});
\end{DoxyVerb}


For a complicated series of async tasks using the auto function makes adding new tasks much easier and makes the code more readable.





\label{_iterator}%
 \subsubsection*{iterator(tasks)}

Creates an iterator function which calls the next function in the array, returning a continuation to call the next one after that. Its also possible to \textquotesingle{}peek\textquotesingle{} the next iterator by doing iterator.\+next().

This function is used internally by the async module but can be useful when you want to manually control the flow of functions in series.

{\bfseries Arguments}


\begin{DoxyItemize}
\item tasks -\/ An array of functions to run, each function is passed a callback it must call on completion.
\end{DoxyItemize}

{\bfseries Example} \begin{DoxyVerb}var iterator = async.iterator([
    function(){ sys.p('one'); },
    function(){ sys.p('two'); },
    function(){ sys.p('three'); }
]);

node> var iterator2 = iterator();
'one'
node> var iterator3 = iterator2();
'two'
node> iterator3();
'three'
node> var nextfn = iterator2.next();
node> nextfn();
'three'
\end{DoxyVerb}






\label{_apply}%
 \subsubsection*{apply(function, arguments..)}

Creates a continuation function with some arguments already applied, a useful shorthand when combined with other control flow functions. Any arguments passed to the returned function are added to the arguments originally passed to apply.

{\bfseries Arguments}


\begin{DoxyItemize}
\item function -\/ The function you want to eventually apply all arguments to.
\item arguments... -\/ Any number of arguments to automatically apply when the continuation is called.
\end{DoxyItemize}

{\bfseries Example} \begin{DoxyVerb}// using apply

async.parallel([
    async.apply(fs.writeFile, 'testfile1', 'test1'),
    async.apply(fs.writeFile, 'testfile2', 'test2'),
]);


// the same process without using apply

async.parallel([
    function(callback){
        fs.writeFile('testfile1', 'test1', callback);
    },
    function(callback){
        fs.writeFile('testfile2', 'test2', callback);
    },
]);
\end{DoxyVerb}


It\textquotesingle{}s possible to pass any number of additional arguments when calling the continuation\+: \begin{DoxyVerb}node> var fn = async.apply(sys.puts, 'one');
node> fn('two', 'three');
one
two
three
\end{DoxyVerb}






\label{_nextTick}%
 \subsubsection*{next\+Tick(callback)}

Calls the callback on a later loop around the event loop. In node.\+js this just calls process.\+next\+Tick, in the browser it falls back to set\+Timeout(callback, 0), which means other higher priority events may precede the execution of the callback.

This is used internally for browser-\/compatibility purposes.

{\bfseries Arguments}


\begin{DoxyItemize}
\item callback -\/ The function to call on a later loop around the event loop.
\end{DoxyItemize}

{\bfseries Example} \begin{DoxyVerb}var call_order = [];
async.nextTick(function(){
    call_order.push('two');
    // call_order now equals ['one','two]
});
call_order.push('one')
\end{DoxyVerb}


\subsection*{Utils}

\label{_memoize}%
 \subsubsection*{memoize(fn, \mbox{[}hasher\mbox{]})}

Caches the results of an async function. When creating a hash to store function results against, the callback is omitted from the hash and an optional hash function can be used.

{\bfseries Arguments}


\begin{DoxyItemize}
\item fn -\/ the function you to proxy and cache results from.
\item hasher -\/ an optional function for generating a custom hash for storing results, it has all the arguments applied to it apart from the callback, and must be synchronous.
\end{DoxyItemize}

{\bfseries Example} \begin{DoxyVerb}var slow_fn = function (name, callback) {
    // do something
    callback(null, result);
};
var fn = async.memoize(slow_fn);

// fn can now be used as if it were slow_fn
fn('some name', function () {
    // callback
});
\end{DoxyVerb}


\label{_unmemoize}%
 \subsubsection*{unmemoize(fn)}

Undoes a memoized function, reverting it to the original, unmemoized form. Comes handy in tests.

{\bfseries Arguments}


\begin{DoxyItemize}
\item fn -\/ the memoized function
\end{DoxyItemize}

\label{_log}%
 \subsubsection*{log(function, arguments)}

Logs the result of an async function to the console. Only works in node.\+js or in browsers that support console.\+log and console.\+error (such as F\+F and Chrome). If multiple arguments are returned from the async function, console.\+log is called on each argument in order.

{\bfseries Arguments}


\begin{DoxyItemize}
\item function -\/ The function you want to eventually apply all arguments to.
\item arguments... -\/ Any number of arguments to apply to the function.
\end{DoxyItemize}

{\bfseries Example} \begin{DoxyVerb}var hello = function(name, callback){
    setTimeout(function(){
        callback(null, 'hello ' + name);
    }, 1000);
};

node> async.log(hello, 'world');
'hello world'
\end{DoxyVerb}






\label{_dir}%
 \subsubsection*{dir(function, arguments)}

Logs the result of an async function to the console using console.\+dir to display the properties of the resulting object. Only works in node.\+js or in browsers that support console.\+dir and console.\+error (such as F\+F and Chrome). If multiple arguments are returned from the async function, console.\+dir is called on each argument in order.

{\bfseries Arguments}


\begin{DoxyItemize}
\item function -\/ The function you want to eventually apply all arguments to.
\item arguments... -\/ Any number of arguments to apply to the function.
\end{DoxyItemize}

{\bfseries Example} \begin{DoxyVerb}var hello = function(name, callback){
    setTimeout(function(){
        callback(null, {hello: name});
    }, 1000);
};

node> async.dir(hello, 'world');
{hello: 'world'}
\end{DoxyVerb}






\label{_noConflict}%
 \subsubsection*{no\+Conflict()}

Changes the value of async back to its original value, returning a reference to the async object. 