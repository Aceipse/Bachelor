\hypertarget{_c_1_2_users_2martin_2_documents_2_git_hub_visual_studio_2_bachelor_2_wis_r_2_wis_r_2_scripts_2angular-touch_8js-example}{}\section{C\+:/\+Users/martin/\+Documents/\+Git\+Hub\+Visual\+Studio/\+Bachelor/\+Wis\+R/\+Wis\+R/\+Scripts/angular-\/touch.\+js}
$<$file name=\char`\"{}index.\+html\char`\"{}$>$ $<$button ng-\/click=\char`\"{}count = count + 1\char`\"{} ng-\/init=\char`\"{}count=0\char`\"{}$>$ Increment $<$/button$>$ count\+: \{\{ count \}\} $<$/file$>$ $<$file name=\char`\"{}script.\+js\char`\"{}$>$ angular.\+module(\textquotesingle{}ng\+Click\+Example\textquotesingle{}, \mbox{[}\textquotesingle{}ng\+Touch\textquotesingle{}\mbox{]}); $<$/file$>$ 


\begin{DoxyCodeInclude}

(\textcolor{keyword}{function}(window, angular, undefined) \{\textcolor{stringliteral}{'use strict'};

\textcolor{comment}{// define ngTouch module}
\textcolor{comment}{/* global -ngTouch */}
var ngTouch = angular.module(\textcolor{stringliteral}{'ngTouch'}, []);

\textcolor{keyword}{function} nodeName\_(element) \{
  \textcolor{keywordflow}{return} angular.lowercase(element.nodeName || (element[0] && element[0].nodeName));
\}

\textcolor{comment}{/* global ngTouch: false */}

ngTouch.factory(\textcolor{stringliteral}{'$swipe'}, [\textcolor{keyword}{function}() \{
  \textcolor{comment}{// The total distance in any direction before we make the call on swipe vs. scroll.}
  var MOVE\_BUFFER\_RADIUS = 10;

  var POINTER\_EVENTS = \{
    \textcolor{stringliteral}{'mouse'}: \{
      start: \textcolor{stringliteral}{'mousedown'},
      move: \textcolor{stringliteral}{'mousemove'},
      end: \textcolor{stringliteral}{'mouseup'}
    \},
    \textcolor{stringliteral}{'touch'}: \{
      start: \textcolor{stringliteral}{'touchstart'},
      move: \textcolor{stringliteral}{'touchmove'},
      end: \textcolor{stringliteral}{'touchend'},
      cancel: \textcolor{stringliteral}{'touchcancel'}
    \}
  \};

  \textcolor{keyword}{function} getCoordinates(event) \{
    var originalEvent = \textcolor{keyword}{event}.originalEvent || event;
    var touches = originalEvent.touches && originalEvent.touches.length ? originalEvent.touches : [
      originalEvent];
    var e = (originalEvent.changedTouches && originalEvent.changedTouches[0]) || touches[0];

    \textcolor{keywordflow}{return} \{
      x: e.clientX,
      y: e.clientY
    \};
  \}

  \textcolor{keyword}{function} getEvents(pointerTypes, eventType) \{
    var res = [];
    angular.forEach(pointerTypes, \textcolor{keyword}{function}(pointerType) \{
      var eventName = POINTER\_EVENTS[pointerType][eventType];
      \textcolor{keywordflow}{if} (eventName) \{
        res.push(eventName);
      \}
    \});
    \textcolor{keywordflow}{return} res.join(\textcolor{charliteral}{' '});
  \}

  \textcolor{keywordflow}{return} \{
    bind: \textcolor{keyword}{function}(element, eventHandlers, pointerTypes) \{
      \textcolor{comment}{// Absolute total movement, used to control swipe vs. scroll.}
      var totalX, totalY;
      \textcolor{comment}{// Coordinates of the start position.}
      var startCoords;
      \textcolor{comment}{// Last event's position.}
      var lastPos;
      \textcolor{comment}{// Whether a swipe is active.}
      var active = \textcolor{keyword}{false};

      pointerTypes = pointerTypes || [\textcolor{stringliteral}{'mouse'}, \textcolor{stringliteral}{'touch'}];
      element.on(getEvents(pointerTypes, \textcolor{stringliteral}{'start'}), \textcolor{keyword}{function}(event) \{
        startCoords = getCoordinates(event);
        active = \textcolor{keyword}{true};
        totalX = 0;
        totalY = 0;
        lastPos = startCoords;
        eventHandlers[\textcolor{stringliteral}{'start'}] && eventHandlers[\textcolor{stringliteral}{'start'}](startCoords, event);
      \});
      var events = getEvents(pointerTypes, \textcolor{stringliteral}{'cancel'});
      \textcolor{keywordflow}{if} (events) \{
        element.on(events, \textcolor{keyword}{function}(event) \{
          active = \textcolor{keyword}{false};
          eventHandlers[\textcolor{stringliteral}{'cancel'}] && eventHandlers[\textcolor{stringliteral}{'cancel'}](event);
        \});
      \}

      element.on(getEvents(pointerTypes, \textcolor{stringliteral}{'move'}), \textcolor{keyword}{function}(event) \{
        \textcolor{keywordflow}{if} (!active) \textcolor{keywordflow}{return};

        \textcolor{comment}{// Android will send a touchcancel if it thinks we're starting to scroll.}
        \textcolor{comment}{// So when the total distance (+ or - or both) exceeds 10px in either direction,}
        \textcolor{comment}{// we either:}
        \textcolor{comment}{// - On totalX > totalY, we send preventDefault() and treat this as a swipe.}
        \textcolor{comment}{// - On totalY > totalX, we let the browser handle it as a scroll.}

        \textcolor{keywordflow}{if} (!startCoords) \textcolor{keywordflow}{return};
        var coords = getCoordinates(event);

        totalX += Math.abs(coords.x - lastPos.x);
        totalY += Math.abs(coords.y - lastPos.y);

        lastPos = coords;

        \textcolor{keywordflow}{if} (totalX < MOVE\_BUFFER\_RADIUS && totalY < MOVE\_BUFFER\_RADIUS) \{
          \textcolor{keywordflow}{return};
        \}

        \textcolor{comment}{// One of totalX or totalY has exceeded the buffer, so decide on swipe vs. scroll.}
        \textcolor{keywordflow}{if} (totalY > totalX) \{
          \textcolor{comment}{// Allow native scrolling to take over.}
          active = \textcolor{keyword}{false};
          eventHandlers[\textcolor{stringliteral}{'cancel'}] && eventHandlers[\textcolor{stringliteral}{'cancel'}](event);
          \textcolor{keywordflow}{return};
        \} \textcolor{keywordflow}{else} \{
          \textcolor{comment}{// Prevent the browser from scrolling.}
          \textcolor{keyword}{event}.preventDefault();
          eventHandlers[\textcolor{stringliteral}{'move'}] && eventHandlers[\textcolor{stringliteral}{'move'}](coords, event);
        \}
      \});

      element.on(getEvents(pointerTypes, \textcolor{stringliteral}{'end'}), \textcolor{keyword}{function}(event) \{
        \textcolor{keywordflow}{if} (!active) \textcolor{keywordflow}{return};
        active = \textcolor{keyword}{false};
        eventHandlers[\textcolor{stringliteral}{'end'}] && eventHandlers[\textcolor{stringliteral}{'end'}](getCoordinates(event), event);
      \});
    \}
  \};
\}]);

\textcolor{comment}{/* global ngTouch: false,}
\textcolor{comment}{  nodeName\_: false}
\textcolor{comment}{*/}

ngTouch.config([\textcolor{stringliteral}{'$provide'}, \textcolor{keyword}{function}($provide) \{
  $provide.decorator(\textcolor{stringliteral}{'ngClickDirective'}, [\textcolor{stringliteral}{'$delegate'}, \textcolor{keyword}{function}($delegate) \{
    \textcolor{comment}{// drop the default ngClick directive}
    $delegate.shift();
    \textcolor{keywordflow}{return} $delegate;
  \}]);
\}]);

ngTouch.directive(\textcolor{stringliteral}{'ngClick'}, [\textcolor{stringliteral}{'$parse'}, \textcolor{stringliteral}{'$timeout'}, \textcolor{stringliteral}{'$rootElement'},
    \textcolor{keyword}{function}($parse, $timeout, $rootElement) \{
  var TAP\_DURATION = 750; \textcolor{comment}{// Shorter than 750ms is a tap, longer is a taphold or drag.}
  var MOVE\_TOLERANCE = 12; \textcolor{comment}{// 12px seems to work in most mobile browsers.}
  var PREVENT\_DURATION = 2500; \textcolor{comment}{// 2.5 seconds maximum from preventGhostClick call to click}
  var CLICKBUSTER\_THRESHOLD = 25; \textcolor{comment}{// 25 pixels in any dimension is the limit for busting clicks.}

  var ACTIVE\_CLASS\_NAME = \textcolor{stringliteral}{'ng-click-active'};
  var lastPreventedTime;
  var touchCoordinates;
  var lastLabelClickCoordinates;


  \textcolor{comment}{// TAP EVENTS AND GHOST CLICKS}
  \textcolor{comment}{//}
  \textcolor{comment}{// Why tap events?}
  \textcolor{comment}{// Mobile browsers detect a tap, then wait a moment (usually ~300ms) to see if you're}
  \textcolor{comment}{// double-tapping, and then fire a click event.}
  \textcolor{comment}{//}
  \textcolor{comment}{// This delay sucks and makes mobile apps feel unresponsive.}
  \textcolor{comment}{// So we detect touchstart, touchcancel and touchend ourselves and determine when}
  \textcolor{comment}{// the user has tapped on something.}
  \textcolor{comment}{//}
  \textcolor{comment}{// What happens when the browser then generates a click event?}
  \textcolor{comment}{// The browser, of course, also detects the tap and fires a click after a delay. This results in}
  \textcolor{comment}{// tapping/clicking twice. We do "clickbusting" to prevent it.}
  \textcolor{comment}{//}
  \textcolor{comment}{// How does it work?}
  \textcolor{comment}{// We attach global touchstart and click handlers, that run during the capture (early) phase.}
  \textcolor{comment}{// So the sequence for a tap is:}
  \textcolor{comment}{// - global touchstart: Sets an "allowable region" at the point touched.}
  \textcolor{comment}{// - element's touchstart: Starts a touch}
  \textcolor{comment}{// (- touchcancel ends the touch, no click follows)}
  \textcolor{comment}{// - element's touchend: Determines if the tap is valid (didn't move too far away, didn't hold}
  \textcolor{comment}{//   too long) and fires the user's tap handler. The touchend also calls preventGhostClick().}
  \textcolor{comment}{// - preventGhostClick() removes the allowable region the global touchstart created.}
  \textcolor{comment}{// - The browser generates a click event.}
  \textcolor{comment}{// - The global click handler catches the click, and checks whether it was in an allowable region.}
  \textcolor{comment}{//     - If preventGhostClick was called, the region will have been removed, the click is busted.}
  \textcolor{comment}{//     - If the region is still there, the click proceeds normally. Therefore clicks on links and}
  \textcolor{comment}{//       other elements without ngTap on them work normally.}
  \textcolor{comment}{//}
  \textcolor{comment}{// This is an ugly, terrible hack!}
  \textcolor{comment}{// Yeah, tell me about it. The alternatives are using the slow click events, or making our users}
  \textcolor{comment}{// deal with the ghost clicks, so I consider this the least of evils. Fortunately Angular}
  \textcolor{comment}{// encapsulates this ugly logic away from the user.}
  \textcolor{comment}{//}
  \textcolor{comment}{// Why not just put click handlers on the element?}
  \textcolor{comment}{// We do that too, just to be sure. If the tap event caused the DOM to change,}
  \textcolor{comment}{// it is possible another element is now in that position. To take account for these possibly}
  \textcolor{comment}{// distinct elements, the handlers are global and care only about coordinates.}

  \textcolor{comment}{// Checks if the coordinates are close enough to be within the region.}
  \textcolor{keyword}{function} hit(x1, y1, x2, y2) \{
    \textcolor{keywordflow}{return} Math.abs(x1 - x2) < CLICKBUSTER\_THRESHOLD && Math.abs(y1 - y2) < CLICKBUSTER\_THRESHOLD;
  \}

  \textcolor{comment}{// Checks a list of allowable regions against a click location.}
  \textcolor{comment}{// Returns true if the click should be allowed.}
  \textcolor{comment}{// Splices out the allowable region from the list after it has been used.}
  \textcolor{keyword}{function} checkAllowableRegions(touchCoordinates, x, y) \{
    \textcolor{keywordflow}{for} (var i = 0; i < touchCoordinates.length; i += 2) \{
      \textcolor{keywordflow}{if} (hit(touchCoordinates[i], touchCoordinates[i + 1], x, y)) \{
        touchCoordinates.splice(i, i + 2);
        \textcolor{keywordflow}{return} \textcolor{keyword}{true}; \textcolor{comment}{// allowable region}
      \}
    \}
    \textcolor{keywordflow}{return} \textcolor{keyword}{false}; \textcolor{comment}{// No allowable region; bust it.}
  \}

  \textcolor{comment}{// Global click handler that prevents the click if it's in a bustable zone and preventGhostClick}
  \textcolor{comment}{// was called recently.}
  \textcolor{keyword}{function} onClick(event) \{
    \textcolor{keywordflow}{if} (Date.now() - lastPreventedTime > PREVENT\_DURATION) \{
      \textcolor{keywordflow}{return}; \textcolor{comment}{// Too old.}
    \}

    var touches = \textcolor{keyword}{event}.touches && \textcolor{keyword}{event}.touches.length ? \textcolor{keyword}{event}.touches : [event];
    var x = touches[0].clientX;
    var y = touches[0].clientY;
    \textcolor{comment}{// Work around desktop Webkit quirk where clicking a label will fire two clicks (on the label}
    \textcolor{comment}{// and on the input element). Depending on the exact browser, this second click we don't want}
    \textcolor{comment}{// to bust has either (0,0), negative coordinates, or coordinates equal to triggering label}
    \textcolor{comment}{// click event}
    \textcolor{keywordflow}{if} (x < 1 && y < 1) \{
      \textcolor{keywordflow}{return}; \textcolor{comment}{// offscreen}
    \}
    \textcolor{keywordflow}{if} (lastLabelClickCoordinates &&
        lastLabelClickCoordinates[0] === x && lastLabelClickCoordinates[1] === y) \{
      \textcolor{keywordflow}{return}; \textcolor{comment}{// input click triggered by label click}
    \}
    \textcolor{comment}{// reset label click coordinates on first subsequent click}
    \textcolor{keywordflow}{if} (lastLabelClickCoordinates) \{
      lastLabelClickCoordinates = null;
    \}
    \textcolor{comment}{// remember label click coordinates to prevent click busting of trigger click event on input}
    \textcolor{keywordflow}{if} (nodeName\_(event.target) === \textcolor{stringliteral}{'label'}) \{
      lastLabelClickCoordinates = [x, y];
    \}

    \textcolor{comment}{// Look for an allowable region containing this click.}
    \textcolor{comment}{// If we find one, that means it was created by touchstart and not removed by}
    \textcolor{comment}{// preventGhostClick, so we don't bust it.}
    \textcolor{keywordflow}{if} (checkAllowableRegions(touchCoordinates, x, y)) \{
      \textcolor{keywordflow}{return};
    \}

    \textcolor{comment}{// If we didn't find an allowable region, bust the click.}
    \textcolor{keyword}{event}.stopPropagation();
    \textcolor{keyword}{event}.preventDefault();

    \textcolor{comment}{// Blur focused form elements}
    \textcolor{keyword}{event}.target && \textcolor{keyword}{event}.target.blur && \textcolor{keyword}{event}.target.blur();
  \}


  \textcolor{comment}{// Global touchstart handler that creates an allowable region for a click event.}
  \textcolor{comment}{// This allowable region can be removed by preventGhostClick if we want to bust it.}
  \textcolor{keyword}{function} onTouchStart(event) \{
    var touches = \textcolor{keyword}{event}.touches && \textcolor{keyword}{event}.touches.length ? \textcolor{keyword}{event}.touches : [event];
    var x = touches[0].clientX;
    var y = touches[0].clientY;
    touchCoordinates.push(x, y);

    $timeout(\textcolor{keyword}{function}() \{
      \textcolor{comment}{// Remove the allowable region.}
      \textcolor{keywordflow}{for} (var i = 0; i < touchCoordinates.length; i += 2) \{
        \textcolor{keywordflow}{if} (touchCoordinates[i] == x && touchCoordinates[i + 1] == y) \{
          touchCoordinates.splice(i, i + 2);
          \textcolor{keywordflow}{return};
        \}
      \}
    \}, PREVENT\_DURATION, \textcolor{keyword}{false});
  \}

  \textcolor{comment}{// On the first call, attaches some event handlers. Then whenever it gets called, it creates a}
  \textcolor{comment}{// zone around the touchstart where clicks will get busted.}
  \textcolor{keyword}{function} preventGhostClick(x, y) \{
    \textcolor{keywordflow}{if} (!touchCoordinates) \{
      $rootElement[0].addEventListener(\textcolor{stringliteral}{'click'}, onClick, \textcolor{keyword}{true});
      $rootElement[0].addEventListener(\textcolor{stringliteral}{'touchstart'}, onTouchStart, \textcolor{keyword}{true});
      touchCoordinates = [];
    \}

    lastPreventedTime = Date.now();

    checkAllowableRegions(touchCoordinates, x, y);
  \}

  \textcolor{comment}{// Actual linking function.}
  \textcolor{keywordflow}{return} \textcolor{keyword}{function}(scope, element, attr) \{
    var clickHandler = $parse(attr.ngClick),
        tapping = \textcolor{keyword}{false},
        tapElement,  \textcolor{comment}{// Used to blur the element after a tap.}
        startTime,   \textcolor{comment}{// Used to check if the tap was held too long.}
        touchStartX,
        touchStartY;

    \textcolor{keyword}{function} resetState() \{
      tapping = \textcolor{keyword}{false};
      element.removeClass(ACTIVE\_CLASS\_NAME);
    \}

    element.on(\textcolor{stringliteral}{'touchstart'}, \textcolor{keyword}{function}(event) \{
      tapping = \textcolor{keyword}{true};
      tapElement = \textcolor{keyword}{event}.target ? \textcolor{keyword}{event}.target : \textcolor{keyword}{event}.srcElement; \textcolor{comment}{// IE uses srcElement.}
      \textcolor{comment}{// Hack for Safari, which can target text nodes instead of containers.}
      \textcolor{keywordflow}{if} (tapElement.nodeType == 3) \{
        tapElement = tapElement.parentNode;
      \}

      element.addClass(ACTIVE\_CLASS\_NAME);

      startTime = Date.now();

      \textcolor{comment}{// Use jQuery originalEvent}
      var originalEvent = \textcolor{keyword}{event}.originalEvent || event;
      var touches = originalEvent.touches && originalEvent.touches.length ? originalEvent.touches : [
      originalEvent];
      var e = touches[0];
      touchStartX = e.clientX;
      touchStartY = e.clientY;
    \});

    element.on(\textcolor{stringliteral}{'touchcancel'}, \textcolor{keyword}{function}(event) \{
      resetState();
    \});

    element.on(\textcolor{stringliteral}{'touchend'}, \textcolor{keyword}{function}(event) \{
      var diff = Date.now() - startTime;

      \textcolor{comment}{// Use jQuery originalEvent}
      var originalEvent = \textcolor{keyword}{event}.originalEvent || event;
      var touches = (originalEvent.changedTouches && originalEvent.changedTouches.length) ?
          originalEvent.changedTouches :
          ((originalEvent.touches && originalEvent.touches.length) ? originalEvent.touches : [originalEvent
      ]);
      var e = touches[0];
      var x = e.clientX;
      var y = e.clientY;
      var dist = Math.sqrt(Math.pow(x - touchStartX, 2) + Math.pow(y - touchStartY, 2));

      \textcolor{keywordflow}{if} (tapping && diff < TAP\_DURATION && dist < MOVE\_TOLERANCE) \{
        \textcolor{comment}{// Call preventGhostClick so the clickbuster will catch the corresponding click.}
        preventGhostClick(x, y);

        \textcolor{comment}{// Blur the focused element (the button, probably) before firing the callback.}
        \textcolor{comment}{// This doesn't work perfectly on Android Chrome, but seems to work elsewhere.}
        \textcolor{comment}{// I couldn't get anything to work reliably on Android Chrome.}
        \textcolor{keywordflow}{if} (tapElement) \{
          tapElement.blur();
        \}

        \textcolor{keywordflow}{if} (!angular.isDefined(attr.disabled) || attr.disabled === \textcolor{keyword}{false}) \{
          element.triggerHandler(\textcolor{stringliteral}{'click'}, [event]);
        \}
      \}

      resetState();
    \});

    \textcolor{comment}{// Hack for iOS Safari's benefit. It goes searching for onclick handlers and is liable to click}
    \textcolor{comment}{// something else nearby.}
    element.onclick = \textcolor{keyword}{function}(event) \{ \};

    \textcolor{comment}{// Actual click handler.}
    \textcolor{comment}{// There are three different kinds of clicks, only two of which reach this point.}
    \textcolor{comment}{// - On desktop browsers without touch events, their clicks will always come here.}
    \textcolor{comment}{// - On mobile browsers, the simulated "fast" click will call this.}
    \textcolor{comment}{// - But the browser's follow-up slow click will be "busted" before it reaches this handler.}
    \textcolor{comment}{// Therefore it's safe to use this directive on both mobile and desktop.}
    element.on(\textcolor{stringliteral}{'click'}, \textcolor{keyword}{function}(event, touchend) \{
      scope.$apply(\textcolor{keyword}{function}() \{
        clickHandler(scope, \{$event: (touchend || event)\});
      \});
    \});

    element.on(\textcolor{stringliteral}{'mousedown'}, \textcolor{keyword}{function}(event) \{
      element.addClass(ACTIVE\_CLASS\_NAME);
    \});

    element.on(\textcolor{stringliteral}{'mousemove mouseup'}, \textcolor{keyword}{function}(event) \{
      element.removeClass(ACTIVE\_CLASS\_NAME);
    \});

  \};
\}]);

\textcolor{comment}{/* global ngTouch: false */}

\textcolor{keyword}{function} makeSwipeDirective(directiveName, direction, eventName) \{
  ngTouch.directive(directiveName, [\textcolor{stringliteral}{'$parse'}, \textcolor{stringliteral}{'$swipe'}, \textcolor{keyword}{function}($parse, $swipe) \{
    \textcolor{comment}{// The maximum vertical delta for a swipe should be less than 75px.}
    var MAX\_VERTICAL\_DISTANCE = 75;
    \textcolor{comment}{// Vertical distance should not be more than a fraction of the horizontal distance.}
    var MAX\_VERTICAL\_RATIO = 0.3;
    \textcolor{comment}{// At least a 30px lateral motion is necessary for a swipe.}
    var MIN\_HORIZONTAL\_DISTANCE = 30;

    \textcolor{keywordflow}{return} \textcolor{keyword}{function}(scope, element, attr) \{
      var swipeHandler = $parse(attr[directiveName]);

      var startCoords, valid;

      \textcolor{keyword}{function} validSwipe(coords) \{
        \textcolor{comment}{// Check that it's within the coordinates.}
        \textcolor{comment}{// Absolute vertical distance must be within tolerances.}
        \textcolor{comment}{// Horizontal distance, we take the current X - the starting X.}
        \textcolor{comment}{// This is negative for leftward swipes and positive for rightward swipes.}
        \textcolor{comment}{// After multiplying by the direction (-1 for left, +1 for right), legal swipes}
        \textcolor{comment}{// (ie. same direction as the directive wants) will have a positive delta and}
        \textcolor{comment}{// illegal ones a negative delta.}
        \textcolor{comment}{// Therefore this delta must be positive, and larger than the minimum.}
        \textcolor{keywordflow}{if} (!startCoords) \textcolor{keywordflow}{return} \textcolor{keyword}{false};
        var deltaY = Math.abs(coords.y - startCoords.y);
        var deltaX = (coords.x - startCoords.x) * direction;
        \textcolor{keywordflow}{return} valid && \textcolor{comment}{// Short circuit for already-invalidated swipes.}
            deltaY < MAX\_VERTICAL\_DISTANCE &&
            deltaX > 0 &&
            deltaX > MIN\_HORIZONTAL\_DISTANCE &&
            deltaY / deltaX < MAX\_VERTICAL\_RATIO;
      \}

      var pointerTypes = [\textcolor{stringliteral}{'touch'}];
      \textcolor{keywordflow}{if} (!angular.isDefined(attr[\textcolor{stringliteral}{'ngSwipeDisableMouse'}])) \{
        pointerTypes.push(\textcolor{stringliteral}{'mouse'});
      \}
      $swipe.bind(element, \{
        \textcolor{stringliteral}{'start'}: \textcolor{keyword}{function}(coords, event) \{
          startCoords = coords;
          valid = \textcolor{keyword}{true};
        \},
        \textcolor{stringliteral}{'cancel'}: \textcolor{keyword}{function}(event) \{
          valid = \textcolor{keyword}{false};
        \},
        \textcolor{stringliteral}{'end'}: \textcolor{keyword}{function}(coords, event) \{
          \textcolor{keywordflow}{if} (validSwipe(coords)) \{
            scope.$apply(\textcolor{keyword}{function}() \{
              element.triggerHandler(eventName);
              swipeHandler(scope, \{$event: \textcolor{keyword}{event}\});
            \});
          \}
        \}
      \}, pointerTypes);
    \};
  \}]);
\}

\textcolor{comment}{// Left is negative X-coordinate, right is positive.}
makeSwipeDirective(\textcolor{stringliteral}{'ngSwipeLeft'}, -1, \textcolor{stringliteral}{'swipeleft'});
makeSwipeDirective(\textcolor{stringliteral}{'ngSwipeRight'}, 1, \textcolor{stringliteral}{'swiperight'});



\})(window, window.angular);
\end{DoxyCodeInclude}
 