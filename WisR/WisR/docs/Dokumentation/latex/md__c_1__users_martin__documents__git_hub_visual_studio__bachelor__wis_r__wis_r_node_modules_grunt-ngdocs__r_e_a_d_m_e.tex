Grunt plugin to create a documentation like \href{http://docs.angularjs.org}{\tt Angular\+J\+S} N\+O\+T\+E\+: this plugin requires Grunt 0.\+4.\+x

A\+T\+T\+E\+N\+T\+I\+O\+N\+: grunt-\/ngdocs 0.\+2+ is for angularjs 1.\+2+ grunt-\/ngdocs 0.\+2.\+5 supports angularjs 1.\+3+ too Please include angular.\+js and angular-\/animate.\+js with the scripts option

\subsection*{Getting Started}

From the same directory as your project\textquotesingle{}s Gruntfile and package.\+json, install this plugin with the following command\+:

{\ttfamily npm install grunt-\/ngdocs -\/-\/save-\/dev}

Once that\textquotesingle{}s done, add this line to your project\textquotesingle{}s Gruntfile\+:


\begin{DoxyCode}
grunt.loadNpmTasks(\textcolor{stringliteral}{'grunt-ngdocs'});
\end{DoxyCode}


A full working example can be found at \href{https://github.com/m7r/grunt-ngdocs-example}{\tt https\+://github.\+com/m7r/grunt-\/ngdocs-\/example}

\subsection*{Config}

Inside your {\ttfamily Gruntfile.\+js} file, add a section named {\itshape ngdocs}. Here\textquotesingle{}s a simple example\+:


\begin{DoxyCode}
ngdocs: \{
  all: [\textcolor{stringliteral}{'src/**/*.js'}]
\}
\end{DoxyCode}


And with many options\+:


\begin{DoxyCode}
ngdocs: \{
  options: \{
    dest: \textcolor{stringliteral}{'docs'},
    scripts: [\textcolor{stringliteral}{'../app.min.js'}],
    html5Mode: \textcolor{keyword}{true},
    startPage: \textcolor{stringliteral}{'/api'},
    title: \textcolor{stringliteral}{"My Awesome Docs"},
    image: \textcolor{stringliteral}{"path/to/my/image.png"},
    imageLink: \textcolor{stringliteral}{"http://my-domain.com"},
    titleLink: \textcolor{stringliteral}{"/api"},
    inlinePartials: \textcolor{keyword}{true},
    bestMatch: \textcolor{keyword}{true},
    analytics: \{
          account: \textcolor{stringliteral}{'UA-08150815-0'}
    \},
    discussions: \{
          shortName: \textcolor{stringliteral}{'my'},
          url: \textcolor{stringliteral}{'http://my-domain.com'},
          dev: \textcolor{keyword}{false}
    \}
  \},
  tutorial: \{
    src: [\textcolor{stringliteral}{'content/tutorial/*.ngdoc'}],
    title: \textcolor{stringliteral}{'Tutorial'}
  \},
  api: \{
    src: [\textcolor{stringliteral}{'src/**/*.js'}, \textcolor{stringliteral}{'!src/**/*.spec.js'}],
    title: \textcolor{stringliteral}{'API Documentation'}
  \}
\}
\end{DoxyCode}


\subsubsection*{Targets}

Each grunt target creates a section in the documentation app.

\paragraph*{src}

\mbox{[}required\mbox{]} List of files to parse for documentation comments.

\paragraph*{title}

\mbox{[}default\mbox{]} \textquotesingle{}A\+P\+I Documentation\textquotesingle{}

Set the name for the section in the documentation app.

\paragraph*{api}

\mbox{[}default\mbox{]} true for target api

Set the sidebar to advanced mode, with sections for modules, services, etc.

\subsubsection*{Options}

\paragraph*{dest}

\mbox{[}default\mbox{]} \textquotesingle{}docs\textquotesingle{}

Folder relative to your Gruntfile where the documentation should be built.

\paragraph*{scripts}

\mbox{[}default\mbox{]} \mbox{[}\textquotesingle{}angular.\+js\textquotesingle{}\mbox{]}

Set which angular.\+js file or addional custom js files are loaded to the app. This allows the live examples to use custom directives, services, etc. The documentation app works with angular.\+js 1.\+2+ and 1.\+3+. If you include your own angular.\+js include angular-\/animate.\+js too.

Possible values\+:


\begin{DoxyItemize}
\item \mbox{[}\textquotesingle{}angular.\+js\textquotesingle{}\mbox{]} use angular and angular-\/animate 1.\+2.\+16 delivered with grunt-\/ngdocs
\item \mbox{[}\textquotesingle{}path/to/file.\+js\textquotesingle{}\mbox{]} file will be copied into the docs, into a {\ttfamily grunt-\/scripts} folder
\item \mbox{[}\textquotesingle{}\href{http://example.com/file.js',}{\tt http\+://example.\+com/file.\+js\textquotesingle{},} \textquotesingle{}\href{https://example.com/file.js',}{\tt https\+://example.\+com/file.\+js\textquotesingle{},} \textquotesingle{}//example.com/file.\+js\textquotesingle{}\mbox{]} reference remote files (eg from a C\+D\+N)
\item \mbox{[}\textquotesingle{}../app.js\textquotesingle{}\mbox{]} reference file relative to the dest folder
\end{DoxyItemize}

\paragraph*{defer\+Load}

\mbox{[}default\mbox{]} false

If you want to use requirejs as loader set this to {\ttfamily true}.

Include \textquotesingle{}js/angular-\/bootstrap.\+js\textquotesingle{}, \textquotesingle{}js/angular-\/bootstrap-\/prettify.\+js\textquotesingle{}, \textquotesingle{}js/docs-\/setup.\+js\textquotesingle{}, \textquotesingle{}js/docs.\+js\textquotesingle{} with requirejs and finally bootstrap the app `angular.bootstrap(document, \mbox{[}\textquotesingle{}docs\+App\textquotesingle{}\mbox{]});`.

\paragraph*{styles}

\mbox{[}default\mbox{]} \mbox{[}\mbox{]}

Copy additional css files to the documentation app

\paragraph*{template}

\mbox{[}default\mbox{]} null

Allow to use your own template. Use the default template at src/templates/index.\+tmpl as reference.

\paragraph*{start\+Page}

\mbox{[}default\mbox{]} \textquotesingle{}/api\textquotesingle{}

Set first page to open.

\paragraph*{html5\+Mode}

\mbox{[}default\mbox{]} false

Whether or not to enable {\ttfamily html5\+Mode} in the docs application. If true, then links will be absolute. If false, they will be prefixed by {\ttfamily \#/}.

\paragraph*{best\+Match}

\mbox{[}default\mbox{]} false

The best matching page for a search query is highlighted and get selected on return. If this option is set to true the best match is shown below the search field in an dropdown menu. Use this for long lists where the highlight is often not visible.

\paragraph*{title}

\mbox{[}default\mbox{]} \char`\"{}name\char`\"{} or \char`\"{}title\char`\"{} field in {\ttfamily pkg}

Title to put on the navbar and the page\textquotesingle{}s {\ttfamily title} attribute. By default, tries to find the title in the {\ttfamily pkg}. If it can\textquotesingle{}t find it, it will go to an empty string.

\paragraph*{title\+Link}

\mbox{[}default\mbox{]} no anchor tag is used

Wraps the title text in an anchor tag with the provided U\+R\+L.

\paragraph*{image}

A U\+R\+L or relative path to an image file to use in the top navbar.

\paragraph*{image\+Link}

\mbox{[}default\mbox{]} no anchor tag is used

Wraps the navbar image in an anchor tag with the provided U\+R\+L.

\paragraph*{nav\+Template}

\mbox{[}default\mbox{]} null

Path to a template of a nav H\+T\+M\+L template to include. The css for it should be that of listitems inside a bootstrap navbar\+:


\begin{DoxyCode}
1 <header class="header">
2   <div class="navbar">
3     <ul class="nav">
4       \{\{links to all the docs pages\}\}
5     </ul>
6     \{\{YOUR\_NAV\_TEMPLATE\_GOES\_HERE\}\}
7   </div>
8 </header>
\end{DoxyCode}
 Example\+: \textquotesingle{}templates/my-\/nav.\+html\textquotesingle{}

The template, if specified, is pre-\/processed using \href{https://github.com/gruntjs/grunt/wiki/grunt.template#grunttemplateprocess}{\tt grunt.\+template}.

\paragraph*{source\+Link}

\mbox{[}default\mbox{]} true

Display \char`\"{}\+View source\char`\"{} link. Possible values are


\begin{DoxyItemize}
\item {\ttfamily true}\+: try to read repository from package.\+json (currently only github is supported)
\item {\ttfamily false}\+: don\textquotesingle{}t display link
\item string\+: template string like {\ttfamily \textquotesingle{}\href{https://internal.server/repo/blob/}{\tt https\+://internal.\+server/repo/blob/}\{\{sha\}\}/\{\{file\}\}\#\+L\{\{codeline\}\}\textquotesingle{}}

available placeholders\+:
\begin{DoxyItemize}
\item {\bfseries file}\+: path and filename current file
\item {\bfseries filename}\+: only filename of current file
\item {\bfseries filepath}\+: directory of current file
\item {\bfseries line}\+: first line of comment
\item {\bfseries codeline}\+: first line {\itshape after} comment
\item {\bfseries version}\+: version read from package.\+json
\item {\bfseries sha}\+: first 7 characters of current git commit
\end{DoxyItemize}
\end{DoxyItemize}

\paragraph*{edit\+Link}

\mbox{[}default\mbox{]} true

Display \char`\"{}\+Improve this doc\char`\"{} link. Same options as for source\+Link.

\paragraph*{edit\+Example}

\mbox{[}default\mbox{]} true

Show Edit Button for examples.

\paragraph*{inline\+Partials}

\mbox{[}default\mbox{]} false

If set to true this option will turn all partials into angular inline templates and place them inside the generated {\ttfamily index.\+html} file. The advantage over lazyloading with ajax is that the documentation will also work on the {\ttfamily \href{file://}{\tt file\+://}} system.

\paragraph*{discussions}

Optional include \href{http://disqus.com}{\tt discussions} in the documentation app.


\begin{DoxyCode}
\{
  shortName: \textcolor{stringliteral}{'my'},
  url: \textcolor{stringliteral}{'http://my-domain.com'},
  dev: \textcolor{keyword}{false}
\}
\end{DoxyCode}


\paragraph*{analytics}

Optional include Google Analytics in the documentation app.


\begin{DoxyCode}
\{
  account: \textcolor{stringliteral}{'UA-08150815-0'}
\}
\end{DoxyCode}


\subsection*{How it works}

The task parses the specified files for doc comments and extracts them into partial html files for the documentation app.

At first run, all necessary files will be copied to the destination folder. After that, only index.\+html, js/docs-\/setup.\+js, and the partials will be overwritten.

Partials that are no longer needed will not be deleted. Use, for example, the grunt-\/contrib-\/clean task to clean the docs folder before creating a distribution build.

After an update of grunt-\/ngdocs you should clean the docs folder too.

A doc comment looks like this\+:

```js /$\ast$$\ast$
\begin{DoxyItemize}
\item directive
\item 
\end{DoxyItemize}