A module that can be used to implement A\+M\+D\textquotesingle{}s define() in Node. This allows you to code to the A\+M\+D A\+P\+I and have the module work in node programs without requiring those other programs to use A\+M\+D.

\subsection*{Usage}

{\bfseries 1)} Update your package.\+json to indicate amdefine as a dependency\+:


\begin{DoxyCode}
1 "dependencies": \{
2     "amdefine": ">=0.1.0"
3 \}
\end{DoxyCode}


Then run {\ttfamily npm install} to get amdefine into your project.

{\bfseries 2)} At the top of each module that uses define(), place this code\+:


\begin{DoxyCode}
1 if (typeof define !== 'function') \{ var define = require('amdefine')(module) \}
\end{DoxyCode}


{\bfseries Only use these snippets} when loading amdefine. If you preserve the basic structure, with the braces, it will be stripped out when using the \href{#optimizer}{\tt Require\+J\+S optimizer}.

You can add spaces, line breaks and even require amdefine with a local path, but keep the rest of the structure to get the stripping behavior.

As you may know, because {\ttfamily if} statements in Java\+Script don\textquotesingle{}t have their own scope, the var declaration in the above snippet is made whether the {\ttfamily if} expression is truthy or not. If Require\+J\+S is loaded then the declaration is superfluous because {\ttfamily define} is already already declared in the same scope in Require\+J\+S. Fortunately Java\+Script handles multiple {\ttfamily var} declarations of the same variable in the same scope gracefully.

If you want to deliver amdefine.\+js with your code rather than specifying it as a dependency with npm, then just download the latest release and refer to it using a relative path\+:

\href{https://github.com/jrburke/amdefine/raw/latest/amdefine.js}{\tt Latest Version}

\subsubsection*{amdefine/intercept}

Consider this very experimental.

Instead of pasting the piece of text for the amdefine setup of a {\ttfamily define} variable in each module you create or consume, you can use {\ttfamily amdefine/intercept} instead. It will automatically insert the above snippet in each .js file loaded by Node.

{\bfseries Warning}\+: you should only use this if you are creating an application that is consuming A\+M\+D style defined()\textquotesingle{}d modules that are distributed via npm and want to run that code in Node.

For library code where you are not sure if it will be used by others in Node or in the browser, then explicitly depending on amdefine and placing the code snippet above is suggested path, instead of using {\ttfamily amdefine/intercept}. The intercept module affects all .js files loaded in the Node app, and it is inconsiderate to modify global state like that unless you are also controlling the top level app.

\paragraph*{Why distribute A\+M\+D-\/style modules via npm?}

npm has a lot of weaknesses for front-\/end use (installed layout is not great, should have better support for the `base\+Url + module\+I\+D + \textquotesingle{}.js\textquotesingle{} style of loading, single file J\+S installs), but some people want a J\+S package manager and are willing to live with those constraints. If that is you, but still want to author in A\+M\+D style modules to get dynamic require(\mbox{[}$\,$\mbox{]}), better direct source usage and powerful loader plugin support in the browser, then this tool can help.

\paragraph*{amdefine/intercept usage}

Just require it in your top level app module (for example index.\+js, server.\+js)\+:


\begin{DoxyCode}
1 require('amdefine/intercept');
\end{DoxyCode}


The module does not return a value, so no need to assign the result to a local variable.

Then just require() code as you normally would with Node\textquotesingle{}s require(). Any .js loaded after the intercept require will have the amdefine check injected in the .js source as it is loaded. It does not modify the source on disk, just prepends some content to the text of the module as it is loaded by Node.

\paragraph*{How amdefine/intercept works}

It overrides the `\+Module.\+\_\+extensions\mbox{[}\textquotesingle{}.js\textquotesingle{}\mbox{]}` in Node to automatically prepend the amdefine snippet above. So, it will affect any .js file loaded by your app.

\subsection*{define() usage}

It is best if you use the anonymous forms of define() in your module\+:


\begin{DoxyCode}
1 define(function (require) \{
2     var dependency = require('dependency');
3 \});
\end{DoxyCode}


or


\begin{DoxyCode}
1 define(['dependency'], function (dependency) \{
2 
3 \});
\end{DoxyCode}


\subsection*{Require\+J\+S optimizer integration. \label{_optimizer}%
}

Version 1.\+0.\+3 of the \href{http://requirejs.org/docs/optimization.html}{\tt Require\+J\+S optimizer} will have support for stripping the `if (typeof define !== \textquotesingle{}function\textquotesingle{})` check mentioned above, so you can include this snippet for code that runs in the browser, but avoid taking the cost of the if() statement once the code is optimized for deployment.

\subsection*{Node 0.\+4 Support}

If you want to support Node 0.\+4, then add {\ttfamily require} as the second parameter to amdefine\+:


\begin{DoxyCode}
1 //Only if you want Node 0.4. If using 0.5 or later, use the above snippet.
2 if (typeof define !== 'function') \{ var define = require('amdefine')(module, require) \}
\end{DoxyCode}


\subsection*{Limitations}

\subsubsection*{Synchronous vs Asynchronous}

amdefine creates a define() function that is callable by your code. It will execute and trace dependencies and call the factory function {\itshape synchronously}, to keep the behavior in line with Node\textquotesingle{}s synchronous dependency tracing.

The exception\+: calling A\+M\+D\textquotesingle{}s callback-\/style require() from inside a factory function. The require callback is called on process.\+next\+Tick()\+:


\begin{DoxyCode}
1 define(function (require) \{
2     require(['a'], function(a) \{
3         //'a' is loaded synchronously, but
4         //this callback is called on process.nextTick().
5     \});
6 \});
\end{DoxyCode}


\subsubsection*{Loader Plugins}

Loader plugins are supported as long as they call their load() callbacks synchronously. So ones that do network requests will not work. However plugins like \href{http://requirejs.org/docs/api.html#text}{\tt text} can load text files locally.

The plugin A\+P\+I\textquotesingle{}s {\ttfamily load.\+from\+Text()} is {\bfseries not supported} in amdefine, so this means transpiler plugins like the \href{https://github.com/jrburke/require-cs}{\tt Coffee\+Script loader plugin} will not work. This may be fixable, but it is a bit complex, and I do not have enough node-\/fu to figure it out yet. See the source for amdefine.\+js if you want to get an idea of the issues involved.

\subsection*{Tests}

To run the tests, cd to {\bfseries tests} and run\+:


\begin{DoxyCode}
1 node all.js
2 node all-intercept.js
\end{DoxyCode}


\subsection*{License}

New B\+S\+D and M\+I\+T. Check the L\+I\+C\+E\+N\+S\+E file for all the details. 