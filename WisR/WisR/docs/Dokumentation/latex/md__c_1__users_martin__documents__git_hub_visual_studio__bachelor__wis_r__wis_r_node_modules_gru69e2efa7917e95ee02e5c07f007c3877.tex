An N\+P\+M wrapper for \href{http://phantomjs.org/}{\tt Phantom\+J\+S}, headless webkit with J\+S A\+P\+I.

\href{https://travis-ci.org/Medium/phantomjs}{\tt !\mbox{[}Build Status\mbox{]}(https\+://travis-\/ci.\+org/\+Medium/phantomjs.\+svg?branch=master)}

\subsection*{Building and Installing }


\begin{DoxyCode}
1 npm install phantomjs
\end{DoxyCode}


Or grab the source and


\begin{DoxyCode}
1 node ./install.js
\end{DoxyCode}


What this installer is really doing is just grabbing a particular \char`\"{}blessed\char`\"{} (by this module) version of Phantom. As new versions of Phantom are released and vetted, this module will be updated accordingly.

The package has been set up to fetch and run Phantom for Mac\+O\+S (darwin), Linux based platforms (as identified by nodejs), and -- as of version 0.\+2.\+0 -- Windows (thanks to \href{https://github.com/domenic}{\tt Domenic Denicola}). If you spot any platform weirdnesses, let us know or send a patch.

\subsection*{Running }


\begin{DoxyCode}
1 bin/phantomjs [phantom arguments]
\end{DoxyCode}


And npm will install a link to the binary in {\ttfamily node\+\_\+modules/.bin} as it is wont to do.

\subsection*{Running via node }

The package exports a {\ttfamily path} string that contains the path to the phantomjs binary/executable.

Below is an example of using this package via node.


\begin{DoxyCode}
1 var path = require('path')
2 var childProcess = require('child\_process')
3 var phantomjs = require('phantomjs')
4 var binPath = phantomjs.path
5 
6 var childArgs = [
7   path.join(\_\_dirname, 'phantomjs-script.js'),
8   'some other argument (passed to phantomjs script)'
9 ]
10 
11 childProcess.execFile(binPath, childArgs, function(err, stdout, stderr) \{
12   // handle results
13 \})
\end{DoxyCode}


\subsection*{Versioning }

The major and minor number tracks the version of Phantom\+J\+S that will be installed. The patch number is incremented when there is either an installer update or a patch build of the phantom binary.

\subsection*{Deciding Where To Get Phantom\+J\+S }

By default, this package will download phantomjs from {\ttfamily \href{https://bitbucket.org/ariya/phantomjs/downloads}{\tt https\+://bitbucket.\+org/ariya/phantomjs/downloads}}. This should work fine for most people.

\subparagraph*{Downloading from a custom U\+R\+L}

If bitbucket is down, or the Great Firewall is blocking bitbucket, you may need to use a download mirror. To set a mirror, set npm config property {\ttfamily phantomjs\+\_\+cdnurl}. Default is ``.


\begin{DoxyCode}
1 npm install phantomjs --phantomjs\_cdnurl=http://cnpmjs.org/downloads
\end{DoxyCode}


Or add property into your {\ttfamily .npmrc} file (\href{https://www.npmjs.org/doc/files/npmrc.html}{\tt https\+://www.\+npmjs.\+org/doc/files/npmrc.\+html})


\begin{DoxyCode}
1 phantomjs\_cdnurl=http://cnpmjs.org/downloads
\end{DoxyCode}


Another option is to use P\+A\+T\+H variable {\ttfamily P\+H\+A\+N\+T\+O\+M\+J\+S\+\_\+\+C\+D\+N\+U\+R\+L}. 
\begin{DoxyCode}
1 PHANTOMJS\_CDNURL=http://cnpmjs.org/downloads npm install phantomjs
\end{DoxyCode}


\subparagraph*{Using Phantom\+J\+S from disk}

If you plan to install phantomjs many times on a single machine, you can install the {\ttfamily phantomjs} binary on P\+A\+T\+H. The installer will automatically detect and use that for non-\/global installs.

\subsection*{A Note on Phantom\+J\+S }

Phantom\+J\+S is not a library for Node\+J\+S. It\textquotesingle{}s a separate environment and code written for node is unlikely to be compatible. In particular Phantom\+J\+S does not expose a Common J\+S package loader.

This is an {\itshape N\+P\+M wrapper} and can be used to conveniently make Phantom available It is not a Node J\+S wrapper.

I have had reasonable experiences writing standalone Phantom scripts which I then drive from within a node program by spawning phantom in a child process.

Read the Phantom\+J\+S F\+A\+Q for more details\+: \href{http://phantomjs.org/faq.html}{\tt http\+://phantomjs.\+org/faq.\+html}

\subsubsection*{Linux Note}

An extra note on Linux usage, from the Phantom\+J\+S download page\+:

\begin{quote}
This package is built on Cent\+O\+S 5.\+8. It should run successfully on Lucid or more modern systems (including other distributions). There is no requirement to install Qt, Web\+Kit, or any other libraries. It is however expected that some base libraries necessary for rendering (Free\+Type, Fontconfig) and the basic font files are available in the system. \end{quote}


\subsection*{Troubleshooting }

\subparagraph*{Installation fails with {\ttfamily spawn E\+N\+O\+E\+N\+T}}

This is N\+P\+M\textquotesingle{}s way of telling you that it was not able to start a process. It usually means\+:


\begin{DoxyItemize}
\item {\ttfamily node} is not on your P\+A\+T\+H, or otherwise not correctly installed.
\item {\ttfamily tar} is not on your P\+A\+T\+H. This package expects {\ttfamily tar} on your P\+A\+T\+H on Linux-\/based platforms.
\end{DoxyItemize}

Check your specific error message for more information.

\subparagraph*{Installation fails with {\ttfamily Error\+: E\+P\+E\+R\+M} or {\ttfamily operation not permitted} or {\ttfamily permission denied}}

This error means that N\+P\+M was not able to install phantomjs to the file system. There are three major reasons why this could happen\+:


\begin{DoxyItemize}
\item You don\textquotesingle{}t have write access to the installation directory.
\item The permissions in the N\+P\+M cache got messed up, and you need to run {\ttfamily npm cache clean} to fix them.
\item You have over-\/zealous anti-\/virus software installed, and it\textquotesingle{}s blocking file system writes.
\end{DoxyItemize}

\subparagraph*{Installation fails with {\ttfamily Error\+: read E\+C\+O\+N\+N\+R\+E\+S\+E\+T} or {\ttfamily Error\+: connect E\+T\+I\+M\+E\+D\+O\+U\+T}}

This error means that something went wrong with your internet connection, and the installer was not able to download the Phantom\+J\+S binary for your platform. Please try again.

\subparagraph*{I tried again, but I get {\ttfamily E\+C\+O\+N\+N\+R\+E\+S\+E\+T} or {\ttfamily E\+T\+I\+M\+E\+D\+O\+U\+T} consistently.}

Do you live in China, or a country with an authoritarian government? We\textquotesingle{}ve seen problems where the G\+F\+W or local I\+S\+P blocks bitbucket, preventing the installer from downloading the binary.

Try visiting the \href{http://cdn.bitbucket.org/ariya/phantomjs/downloads}{\tt the download page} manually. If that page is blocked, you can try using a different C\+D\+N with the {\ttfamily P\+H\+A\+N\+T\+O\+M\+J\+S\+\_\+\+C\+D\+N\+U\+R\+L} env variable described above.

\subparagraph*{I am behind a corporate proxy that uses self-\/signed S\+S\+L certificates to intercept encrypted traffic.}

You can tell N\+P\+M and the Phantom\+J\+S installer to skip validation of ssl keys with N\+P\+M\textquotesingle{}s \href{https://www.npmjs.org/doc/misc/npm-config.html#strict-ssl}{\tt strict-\/ssl} setting\+:


\begin{DoxyCode}
1 npm set strict-ssl false
\end{DoxyCode}


W\+A\+R\+N\+I\+N\+G\+: Turning off {\ttfamily strict-\/ssl} leaves you vulnerable to attackers reading your encrypted traffic, so run this at your own risk!

\subparagraph*{I tried everything, but my network is b0rked. What do I do?}

If you install Phantom\+J\+S manually, and put it on P\+A\+T\+H, the installer will try to use the manually-\/installed binaries.

\subparagraph*{I\textquotesingle{}m on Debian or Ubuntu, and the installer failed because it couldn\textquotesingle{}t find {\ttfamily node}}

Some Linux distros tried to rename {\ttfamily node} to {\ttfamily nodejs} due to a package conflict. This is a non-\/portable change, and we do not try to support this. The \href{https://github.com/joyent/node/wiki/Installing-Node.js-via-package-manager#ubuntu-mint-elementary-os}{\tt official documentation} recommends that you run {\ttfamily apt-\/get install nodejs-\/legacy} to symlink {\ttfamily node} to {\ttfamily nodejs} on those platforms, or many Node\+J\+S programs won\textquotesingle{}t work properly.

\subsection*{Contributing }

Questions, comments, bug reports, and pull requests are all welcome. Submit them at \href{https://github.com/Obvious/phantomjs/}{\tt the project on Git\+Hub}. If you haven\textquotesingle{}t contributed to an \href{http://github.com/Obvious/}{\tt Obvious} project before please head over to the \href{https://github.com/Obvious/open-source#note-to-external-contributors}{\tt Open Source Project} and fill out an O\+C\+L\+A (it should be pretty painless).

Bug reports that include steps-\/to-\/reproduce (including code) are the best. Even better, make them in the form of pull requests.

\subsection*{Author }

\href{https://github.com/dpup}{\tt Dan Pupius} (\href{http://pupius.co.uk}{\tt personal website}), supported by \href{http://obvious.com/}{\tt The Obvious Corporation}.

\subsection*{License }

Copyright 2012 \href{http://obvious.com/}{\tt The Obvious Corporation}.

Licensed under the Apache License, Version 2.\+0. See the top-\/level file {\ttfamily L\+I\+C\+E\+N\+S\+E.\+txt} and (\href{http://www.apache.org/licenses/LICENSE-2.0}{\tt http\+://www.\+apache.\+org/licenses/\+L\+I\+C\+E\+N\+S\+E-\/2.\+0}). 