Java\+Script codecoverage tool for Grunt

\subsection*{Getting Started}

This plugin requires Grunt $\sim$0.4.\+1

Install this grunt plugin next to your project\textquotesingle{}s https\+://github.com/cowboy/grunt/blob/master/docs/getting\+\_\+started.\+md \char`\"{}\+Gruntfile.\+js\char`\"{} with\+: {\ttfamily npm install grunt-\/istanbul}

Then add this line to your project\textquotesingle{}s {\ttfamily Gruntfile.\+js} gruntfile\+:


\begin{DoxyCode}
1 grunt.loadNpmTasks('grunt-istanbul');
\end{DoxyCode}


\subsection*{Documentation}

To use this grunt-\/istanbul plugin, register a grunt task to run the following\+:


\begin{DoxyEnumerate}
\item Instrument your source code
\item Run your test suite against your instrumented source code
\item \hyperlink{class_store}{Store} your coverage results
\item Make the report
\end{DoxyEnumerate}

For step 2, an environment variable can be used to determine which path to use for loading the source code to run the tests against. For example, when you normally run your tests you want them to point directly at your source code. But when you run your instanbul code coverage task you want your tests to point at your instrumented source code. The {\ttfamily grunt-\/env} plugin can be used for setting an environment variable in a grunt task. Here\textquotesingle{}s an example solution that solves this problem using {\ttfamily grunt-\/env} and {\ttfamily grunt-\/mocha-\/test}\+:


\begin{DoxyCode}
1 // in Gruntfile.js
2 module.exports = function (grunt) \{
3 
4   grunt.initConfig(\{
5     env: \{
6       coverage: \{
7         APP\_DIR\_FOR\_CODE\_COVERAGE: '../test/coverage/instrument/app/'
8       \}
9     \},
10     instrument: \{
11       files: 'app/*.js',
12       options: \{
13         lazy: true,
14         basePath: 'test/coverage/instrument/'
15       \}
16     \},
17     mochaTest: \{
18       options: \{
19         reporter: 'spec'
20       \},
21       src: ['test/*.js']
22     \},
23     storeCoverage: \{
24       options: \{
25         dir: 'test/coverage/reports'
26       \}
27     \},
28     makeReport: \{
29       src: 'test/coverage/reports/**/*.json',
30       options: \{
31         type: 'lcov',
32         dir: 'test/coverage/reports',
33         print: 'detail'
34       \}
35     \}
36   \});
37 
38   grunt.registerTask('coverage', ['env:coverage', 'instrument', 'mochaTest',
39     'storeCoverage', 'makeReport']);
40 \};
\end{DoxyCode}
 
\begin{DoxyCode}
1 // require\_helper.js
2 module.exports = function (path) \{
3   return require((process.env.APP\_DIR\_FOR\_CODE\_COVERAGE || '../app/') + path);
4 \};
\end{DoxyCode}
 
\begin{DoxyCode}
1 // using requireHelper in a test
2 var requireHelper = require('../require\_helper');
3 var formValidator = requireHelper('form\_validator');
\end{DoxyCode}


You can also pass an {\ttfamily instrumenter} argument to the instrument {\ttfamily options} as well as any other arguments that your instrumenter takes.


\begin{DoxyCode}
1 // in Gruntfile.js
2 module.exports = function (grunt) \{
3  var isparta = require('isparta');
4   grunt.initConfig(\{
5     instrument: \{
6       files: 'app/*.es6',
7       options: \{
8         lazy: true,
9         basePath: 'test/coverage/instrument/'
10         babel: \{ignore: false, experimental: true, extensions: ['.es6']\},
11         instrumenter: isparta.Instrumenter
12       \}
13     \}
14   \});
15 \};
\end{DoxyCode}


If you want to specify a current working directory, you can specify a path the cwd {\ttfamily options} \+:


\begin{DoxyCode}
1 // in Gruntfile.js
2 module.exports = function (grunt) \{
3  var isparta = require('isparta');
4   grunt.initConfig(\{
5     instrument: \{
6       files: '**/*.es6',
7       options: \{
8         cwd: 'app/'
9         lazy: true,
10         basePath: 'test/coverage/instrument/'
11         babel: \{ignore: false, experimental: true, extensions: ['.es6']\}
12       \}
13     \}
14   \});
15 \};
\end{DoxyCode}


Also, checkout the example Gruntfile.\+js in this repo (note that you do not need to implement the {\ttfamily reload\+Tasks} task in this example)\+: \href{https://github.com/taichi/grunt-istanbul/blob/master/Gruntfile.js#69}{\tt Gruntfile.\+js}

\subsubsection*{more examples}


\begin{DoxyItemize}
\item \href{http://www.gregjopa.com/2014/02/testing-and-code-coverage-with-node-js-apps/}{\tt Testing and Code Coverage With Node.\+js Apps}
\begin{DoxyItemize}
\item \href{https://github.com/gregjopa/express-app-testing-demo}{\tt gregjopa/express-\/app-\/testing-\/demo}
\end{DoxyItemize}
\end{DoxyItemize}

\subsection*{Contributing}

In lieu of a formal styleguide, take care to maintain the existing coding style. Add unit tests for any new or changed functionality. Lint and test your code using \href{https://github.com/cowboy/grunt}{\tt grunt}.

\subsection*{Release History}

\+\_\+(\+Nothing yet)\+\_\+

\subsection*{License}

Copyright (c) 2014 taichi Licensed under the M\+I\+T license. 